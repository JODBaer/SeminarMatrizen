%
% chapter.tex -- Kapitel über Polynome
%
% (c) 2021 Prof Dr Andreas Müller, OST Ostschweizer Fachhochschule
%
\chapter{Polynome
\label{buch:chapter:polynome}}
\lhead{Polynome}
Ein {\em Polynom} ist ein Ausdruck der Form
\index{Polynom}%
\begin{equation}
p(X) = a_nX^n+a_{n-1}X^{n-1} + \cdots a_2X^2 + a_1X + a_0.
\label{buch:eqn:polynome:polynom}
\end{equation}
Ursprünglich stand das Symbol $X$ als Platzhalter für eine Zahl.
Die Polynomgleichung $Y=p(X)$ drückt dann einen Zusammenhang zwischen
den Grössen $X$ und $Y$ aus.
Zum Beispiel drückt
\begin{equation}
H = -\frac12gT^2 + v_0T +h_0 = p(T)
\label{buch:eqn:polynome:beispiel}
\end{equation}
im Schwerefeld der Erde nahe der Oberfläche einen Zusammenhang
zwischen der Zeit $T$ und der Höhe $H$ eines frei fallenden Körpers aus.
Setzt man einen Wert für $T$ in \eqref{buch:eqn:polynome:beispiel} ein,
erhält man den zugehörigen Wert für $H$.
Man stellt sich hier also vor, dass $T$ eigentlich eine Zahl ist und dass
\eqref{buch:eqn:polynome:polynom}
nur ein ``unfertiger'' Ausdruck oder ein ``Programm'' für eine Berechnung
ist.
In dieser eher arithmetischen Sichtweise ist es aber eigentlich egal, dass in
\index{arithmetische Sichtweise}%
\eqref{buch:eqn:polynome:polynom} nur einfache Multiplikationen und
Additionen vorkommen.
In einem Programm könnten ja auch beliebig komplizierte Operationen
verwendet werden, warum also diese Beschränkung?

Für die nachfolgenden Betrachtungen stellen wir uns $X$ daher nicht
mehr einfach als einen Platzhalter für eine Zahl vor, sondern als ein neues
algebraisches Objekt, für das man die Rechenregeln erst noch definieren muss.
In diesem Kapitel sollen die Regeln zum Beispiel sicherstellen,
dass man mit Polynomen so rechnen kann, wie wenn $X$ eine Zahl wäre.
Es sollen also zum Beispiel die Regeln
\begin{align}
aX&=Xa
&
(a+b)X&=aX+bX
&
a+X &= X+a
\label{buch:eqn:polynome:basic}
\end{align}
gelten.
In dieser algebraischen Sichtweise können je nach den gewählten algebraischen
Rechenregeln für $X$ interessante rechnerische Strukturen abgebildet werden.
\index{algebraische Sichtweise}%
Ziel dieses Kapitels ist zu zeigen, wie man die Rechenregeln für $X$
mit Hilfe von Matrizen allgemein darstellen kann.
Diese Betrachtungsweise wird später in Anwendungen ermöglichen,
handliche Realisierungen für das Rechnen mit Grössen zu finden,
die polynomielle Gleichungen erfüllen.
Ebenso sollen in späteren Kapiteln die Regeln
\eqref{buch:eqn:polynome:basic}
erweitert oder abgelöst werden um weitere Anwendungen zu erschliessen.

Bei der Auswahl der zusätzlichen algebraischen Regeln muss man sehr
vorsichtig vorgehen.
Nimmt man zum Beispiel an, dass man durch $X$ teilen kann, dann würde
dies in der arithmetischen Sichtweise bereits ausschliessen, dass man
für $X$ die Zahl $0$ einsetzen kann.
Aber auch eine Regel wie $X^2 \ge 0$, die für alle reellen Zahlen gilt,
würde die Anwendungsmöglichkeiten zu stark einschränken.
Es gibt zwar keine reelle Zahl, die man in das Polynom $p(X)=X^2+1$
einsetzen könnte, so dass es den Wert $0$ annimmt.
Man könnte $X$ aber als ein neues Objekt ausserhalb von $\mathbb{R}$
betrachten, welches die Gleichung $X^2+1=0$ erfüllt.
In den komplexen Zahlen $\mathbb{C}$ gibt es mit der imaginären
Einheit $i\in\mathbb{C}$ tatsächlich ein Zahl mit der Eigenschaft
$i^2=-1$ und damit eine Objekt, welches die Ungleichung $X^2\ge 0$
verletzt.

Für das Symbol $X$ sollen also die ``üblichen'' Rechenregeln gelten.
Dies ist natürlich nur sinnvoll, wenn man auch mit den Koeffizienten
$a_0,\dots,a_n$ rechnen kann.
Sie müssen also Elemente einer
algebraischen Struktur sein, in der mindestens die Addition und die
Multiplikation definiert sind.
Die ganzen Zahlen $\mathbb{Z}$ kommen dafür in Frage, aber auch
die rationalen oder reellen Zahlen $\mathbb{Q}$ und $\mathbb{R}$.
Man kann sogar noch weiter gehen: man kann als Koeffizienten auch
Vektoren oder sogar Matrizen zulassen.
Polynome können addiert werden, indem die Koeffizienten addiert werden,
und sie können mit Skalaren aus dem Koeffizentenkörper multipliziert werden.
Polynome können aber auch multipliziert werden, was auf die Faltung
der Koeffizienten hinausläuft:
\begin{align}
p(X) &= a_nX^n + a_{n-1}X^{n-1} + \dots + a_1X + a_0
\notag
\\
q(X) &= b_mX^m + b_{m-1}X^{m-1} + \dots + b_1X + b_0
\notag
\\
p(X) q(X) &=
a_{n}b_{m}X^{n+m}
+
(a_{n}b_{m-1}+a_{n-1}b_{m})X^{n+m-1}
+
\ldots
+
(a_1b_0+a_0b_1)X
+
a_0b_0
\label{buch:eqn:polynome:faltung}
\\
&=
\sum_{i + j = k}a_ib_j X^k.
\notag
\end{align}
Dies ist aber nur möglich, wenn die Koeffizienten selbst miteinander
multipliziert werden können, wenn also die Koeffizienten mindestens
Elemente einer Algebra sind.

%
% definitionen.tex -- Definition für das Kapitel über Polynome
%
% (c) 2021 Prof Dr Andreas Müller, OST Ostschweizer Fachhochschule
%
\section{Definitionen
\label{buch:section:polynome:definitionen}}
\rhead{Definitionen}
In diesem Abschnitt stellen wir einige grundlegende Definitionen für das
Rechnen mit Polynomen zusammen.

%
% Skalare
%
\subsection{Polynome
\label{buch:subsection:polynome:polynome}}
Wie schon in der Einleitung angedeutet sind Polynome nur dann sinnvoll,
wenn man mit den Koeffizienten gewisse Rechenoperationen durchführen kann.
Wir brauchen mindestens die Möglichkeit, Koeffizienten zu addieren.
Wenn wir uns vorstellen, dass wir $X$ durch eine Zahl ersetzen können,
dann brauchen wir zusätzlich die Möglichkeit, einen Koeffizienten mit einer
Zahl zu multiplizieren.

Die Struktur, die wir hier beschrieben haben, hängt davon ab, was wir uns
unter einer ``Zahl'' vorstellen.
Wir bezeichnen die Menge, aus der die ``Zahlen'' kommen können mit $R$ und
nennen sie die Menge der Skalare.
\index{Skalar}%
Wenn wir uns vorstellen, dass man die Elemente von $R$ an Stelle von $X$
in das Polynom einsetzen kann, dann muss es möglich sein, in $R$ zu
Multiplizieren und zu Addieren, und es müssen die üblichen Rechenregeln
der Algebra gelten, $R$ muss also ein Ring sein.
\index{Ring}%
Wir werden im folgenden zusätzlich voraussetzen, dass $R$ sogar kommutativ
ist und eine $1$ hat.

\begin{definition}
Sei $R$ ein Ring.
Die Menge
\[
R[X]
=
\{
p(X) = a_nX^n+a_{n-1}X^{n-1} + \dots a_1X+a_0 \mid a_k\in R, n\in\mathbb{N}
\}
\]
heisst die Menge der {\em Polynome} mit Koeffizienten in $R$
oder
{\em Polynome über} $R$.
\index{Polynome über $R$}%
Polynome können addiert werden, indem Koeffizienten mit gleichem Index
addiert werden:
\begin{align*}
p(X) &= a_nX^n + a_{n-1}X^{n-1} + \dots + a_1X + a_0\\
q(X) &= b_nX^n + b_{n-1}X^{n-1} + \dots + b_1X + b_0\\
p(X)+q(X)
&=
(a_n+b_n)X^n
+
(a_{n-1}+b_{n-1})X^{n-1}
+
\dots
+
(a_1+b_1)X
+
(a_0+b_0)
\end{align*}
Die Multiplikation ist durch die Formel~\eqref{buch:eqn:polynome:faltung}
definiert.
\end{definition}

Ein Polynom heisst {\em normiert} oder auch {\em monisch}, wenn der
\index{Polynom!normiert}%
\index{normiertes Polynom}%
\index{Polynom!monisch}%
\index{normiertes Polynom}
höchste Koeffizient oder auch {\em Leitkoeffizient} des Polynoms $1$ ist,
also $a_n=1$.
\index{Leitkoeffizient}%
Wenn man in $R$ durch $a_n$ dividieren kann, dann kann man aus dem Polynom
$p(X)=a_nX^n+\dots$ mit Leitkoeffizient $a_n$ das normierte Polynom
\[
\frac{1}{a_n}p(X) = \frac{1}{a_n}(a_nX^n + \dots + a_0)=
X^n + \frac{a_{n-1}}{a_n}X^{n-1} + \dots + \frac{a_0}{a_n}
\]
machen.
Man sagt auch, das Polynom $p(X)$ wurde {\em normiert}.
Wenn $R$ ein Körper ist, ist die Normierung immer möglich.

Die Tatsache, dass zwei  Polynome nicht gleich viele von $0$ verschiedene Koeffizienten haben müssen,
verkompliziert die Beschreibung der Rechenoperationen ein wenig.
Wir werden daher im Folgenden oft für ein Polynom
\[
p(X)
=
a_nX^n + a_{n-1}X^{n-1} + \dots a_1X+a_0
\]
annehmen, dass alle Koeffizienten $a_{n+1},a_{n+2},\dots$ implizit mit
Wert $0$ definiert sind.
Wir werden uns also erlauben,
\[
p(X)
=
\sum_{k}a_kX^k
=
\sum_{k=0}^\infty a_kX^k
\]
zu schreiben, wobei in der ersten Form das Summenzeichen bedeuten soll,
dass nur über diejenigen Indizes $k$ summiert wird, für die $a_k$
definiert ist.
\label{summenzeichenkonvention}

Die Menge $R[X]$ aller Polynome über $R$ mit den beschriebenen
Operationen ist ein Ring. 
Das Distributivgesetz
\index{Distributivgesetz}%
\[
p(X)(u(X)+v(X)) = p(X)u(X) + p(X)v(X)
\qquad
(p(X)+q(X)) u(X) = p(X)u(X) + q(X)u(X)
\]
zum Beispiel sagt ja nichts anderes, als dass man ausmultiplizieren
kann.
\index{ausmultiplizieren}%
Oder die Assoziativgesetze
\begin{align*}
p(X)+q(X)+r(X)
&=
(p(X)+q(X))+r(X)
=
p(X)+(q(X)+r(X))
\\
p(X)q(X)r(X)
&=
(p(X)q(X))r(X)
=
p(X)(q(X)r(X))
\end{align*}
für die Multiplikation besagen, dass es keine Rolle spielt, in welcher
Reihenfolge man die Additionen oder Multiplikationen ausführt.

%
% Der Grad eines Polynoms
%
\subsection{Grad
\label{buch:subsection:polynome:grad}}

\begin{definition}
Der {\em Grad} eines Polynoms $p(X)$ ist die höchste Potenz von $X$, die im
Polynom vorkommt.
\index{Grad eines Polynoms}%
Das Polynom
\[
p(X) = a_nX^n + a_{n-1}X^{n-1}+\dots a_1X + a_0
\]
hat den Grad $n$, wenn $a_n\ne 0$ ist.
Der Grad von $p$ wird mit $\deg p$ bezeichnet.
Das konstante Polynom $p(X)=a_0$ mit $a_0\ne 0$ hat den Grad $0$.
Der Grad des Nullpolynoms $p(X)=0$ ist definiert als
$-\infty$.
\end{definition}

Der Grad eines Polynoms ist sinnvoll in dem Sinn, dass er sich mit
den Rechenoperationen gut verträgt.
Damit lässt sich weiter unten auch die spezielle Wahl des Grades
des Nullpolynoms begründen.
Es gelten nämlich die folgenden Rechenregeln.

\begin{lemma}
\label{lemma:rechenregelnfuerpolynomgrad}
Sind $p$ und $q$ Polynome mit Koeffizienten in $R$ und $0\ne \lambda\in R$,
dann gilt
\begin{align}
\deg(pq) &\le \deg p + \deg q
\label{buch:eqn:polynome:gradsumme}
\\
\deg(p+q) &\le \max(\deg p, \deg q)
\label{buch:eqn:polynome:gradprodukt}
\\
\deg(\lambda p) &\le \deg p.
\label{buch:eqn:polynome:gradskalar}
\end{align}
\end{lemma}

Die Formel \eqref{buch:eqn:polynome:gradskalar} ist eigentlich
ein Spezialfall von \eqref{buch:eqn:polynome:gradsumme}.
Die Zahl $\lambda\in R$ kann man als Polynom vom Grad $0$ betrachten,
wofür natürlich \eqref{buch:eqn:polynome:gradsumme} gilt, also
$\deg(\lambda p) \le \deg\lambda + \deg p$.

\begin{proof}[Beweis]
Wir schreiben die Polynome wieder in der Form
\[
\begin{aligned}
p(X) &= a_nX^n + a_{n-1}X^{n-1} + \dots + a_1X + a_0&&\Rightarrow&\deg p&=n\\
q(X) &= b_mX^m + b_{m-1}X^{m-1} + \dots + b_1X + b_0&&\Rightarrow&\deg q&=m.
\end{aligned}
\]
Dann kann der höchste Koeffizient in der Summe $p+q$ nicht ``weiter oben''
sein als die grössere von den beiden Zahlen $n$ und $m$ angibt, dies
beweist \eqref{buch:eqn:polynome:gradsumme}.
Ebenso kann der höchste Koeffizient im Produkt nach der
Formel~\eqref{buch:eqn:polynome:faltung} nicht ``weiter oben'' als bei
$n+m$ liegen, dies beweist
beweist \eqref{buch:eqn:polynome:gradprodukt}.
In einem Ring mit Nullteilern
(Siehe Definition~\ref{buch:grundlagen:def:nullteiler})
könnte es passieren, dass $a_nb_m=0$ ist, d.~h.~es ist durchaus möglich,
dass der Grad kleiner ist.
Schliesslich kann der höchste Koeffizient von $\lambda p(X)$ nicht grösser
als der höchste Koeffizient von $p(X)$ sein, was
\eqref{buch:eqn:polynome:gradskalar} beweist.
\end{proof}

In einem nullteilerfreien Ring gelten die Rechenregeln für den Grad exakt:

\begin{lemma}
Sei $R$ ein nullteilerfreier Ring und $p$ und $q$ Polynome über $R$
und $0\ne \lambda\in R$.
Dann gilt
\begin{align}
\deg(pq) &= \deg p + \deg q
\label{buch:eqn:polynome:gradsummeexakt}
\\
\deg(p+q) &\le \max(\deg p, \deg q)
\label{buch:eqn:polynome:gradproduktexakt}
\\
\deg(\lambda p) &= \deg p.
\label{buch:eqn:polynome:gradskalarexakt}
\end{align}
\end{lemma}

\begin{proof}[Beweis]
Der Fall, dass der höchste Koeffizient verschwindet, weil $a_n$, $b_m$
oder $\lambda$ Nullteiler sind, kann unter den gegebenen Voraussetzungen
nicht eintreten, daher werden die in
Lemma~\ref{lemma:rechenregelnfuerpolynomgrad} gefunden Ungleichungen
für Produkte exakt.
\end{proof}

Die Gleichung
\eqref{buch:eqn:polynome:gradskalarexakt}
kann im Fall $\lambda=0$ natürlich nicht gelten.
Betrachten wir $\lambda$ wieder als ein Polynom, dann folgt aus
\eqref{buch:eqn:polynome:gradsummeexakt}, dass
\[
\begin{aligned}
\lambda&\ne 0  &&\Rightarrow& \deg (\lambda p) &= \deg\lambda + \deg p = 0+\deg p
\\
\lambda&=0     &&\Rightarrow& \deg (0 p) &= \deg 0 + \deg p = \deg 0
\end{aligned}
\]
Diese Gleichung kann also nur aufrechterhalten werden, wenn die ``Zahl'' $\deg 0$ die Eigenschaft besitzt, dass man immer noch $\deg 0$ bekommt,
wenn man irgend eine Zahl $\deg p$ hinzuaddiert. Wenn also
\[\deg 0 + \deg p = \deg 0 \qquad \forall \deg p \in \mathbb Z\]
gilt.
So eine Zahl gibt es in den ganzen Zahlen nicht.
Wenn man zu einer ganzen Zahl eine andere ganze Zahl hinzuaddiert, ändert sich fast immer etwas.
Man muss daher $\deg 0 = -\infty$ setzen und festlegen, dass
$-\infty + n = -\infty$ für beliebige ganze Zahlen $n$ gilt.

\begin{definition}
\label{buch:def:definitionen:polynomfilterung}
Die Polynome vom Grad $\le n$ mit Koeffizienten in $R$
bilden die Teilmenge
\[
R^{(n)}[X]
=
\{ p\in R[X] \mid \deg p \le n\}.
\]
Die Mengen $R^{(n)}[X]$ bilden eine {\em Filtrierung} des Polynomrings
$R[X]$, d.~h.~sie sind ineinander geschachtelt
\[
\arraycolsep=4pt
\begin{array}{ccccccccccccccc}
R^{(-\infty)}[X] & \subset
	& R^{(0)}[X] & \subset
		& R^{(1)}[X] & \subset & \dots & \subset
			& R^{(k)}[X] & \subset
				& R^{(k+1)}[X] & \subset & \dots & \subset
					& R[X]\\[3pt]
\bigg\| &
	&\bigg\| &
		&\bigg\| & & &
			&&
				&& & &
					&
\\[3pt]
\{0\} & \subset
	& R & \subset
		& \{a_1X+a_0 \mid a_k\in R\} & \subset & \dots &
\end{array}
\]
und ihre Vereinigung ist $R[X]$.
\end{definition}

Die Formeln für den Grad können wir auch mit den Mengen $R^{(k)}[X]$
ausdrücken:
\begin{align*}
\deg (p+q) &\le \max(\deg p, \deg q)
&&\Rightarrow&
R^{(k)}+R^{(l)}
&\subset R^{(\max(k,l))}
=
R^{(k)}[X] \cup R^{(l)}[X].
\\
\deg (p\cdot q)&=\deg p+\deg q
&&\Rightarrow&
R^{(k)}[X] \cdot R^{(l)}[X]
&=
R^{(k+l)}[X].
\end{align*}


%
% Abschnitt über Teilbarkeit
%
\subsection{Teilbarkeit
\label{buch:subsection:polynome:teilbarkeit}}
Im Ring der ganzen Zahlen sind nicht alle Divisionen ohne Rest
ausführbar, so entsteht das Konzept der Teilbarkeit.
Der Divisionsalgorithmus, den man in der Schule lernt, liefert
\index{Divisionsalgorithmus}%
zu beliebigen ganzen Zahlen $a,b\in\mathbb{Z}$ den Quotienten
$q$ und den Rest $r$ derart, dass $a=qb+r$.
Der Algorithmus basiert auf der Zehnersystemdarstellung 
\begin{align*}
a &= a_n10^{n} + a_{n-1}10^{n-1} + \dots + a_110^{1} + a_0
\\
b &= b_m10^{n} + b_{m-1}10^{n-1} + \dots + b_110^{1} + b_0
\end{align*}
und ermittelt den Quotienten, indem er mit den einzelnen Stellen
$a_k$ und $b_k$ arbeitet.
Er ist also eigentlich ein Algorithmus für die Polynome
\begin{align*}
a &= a_nX^{n} + a_{n-1}X^{n-1} + \dots + a_1X^{1} + a_0
\\
b &= b_mX^{n} + b_{m-1}X^{n-1} + \dots + b_1X^{1} + b_0,
\end{align*}
mit dem einzigen Unterschied, dass statt mit $X$ mit der festen Zahl $X=10$
gearbeitet wird.
Der Divisionsalgorithmus für Polynome lässt sich aber leicht
rekonstruieren.

\subsubsection{Polynomdivision}
Wir zeigen den Polynomdivisionsalgorithmus an einem konkreten Beispiel.
\index{Polynomdivision}%
Gesucht sind Quotient $q\in \mathbb{Z}[X]$ und Rest $r\in\mathbb{Z}[X]$
der beiden Polynome
\begin{equation}
\begin{aligned}
a(X) &= X^4 - X^3 -7X^2 + X + 6\\
b(X) &= X^2+X+1,
\end{aligned}
\label{buch:polynome:eqn:divisionsaufgabe}
\end{equation}
für die also gilt $a=bq+r$.
Die Division ergibt
\[
\arraycolsep=1.4pt
\begin{array}{rcrcrcrcrcrcrcrcrcrcr}
X^4&-& X^3&-&7X^2&+& X&+&6&:&X^2&+&X&+&1&=&X^2&-&2X&-&6=q\\
\llap{$-($}X^4&+& X^3&+& X^2\rlap{$)$}& &  & & & &   & & & & & &   & &  & & \\ \cline{1-5}
   &-&2X^3&-&8X^2&+& X& & & &   & & & & & &   & &  & & \\
   &\llap{$-($}-&2X^3&-&2X^2&-&2X\rlap{$)$}& & & &   & & & & & &   & &  & & \\ \cline{2-7}
   & &    &-&6X^2&+&3X&+&6& &   & & & & & &   & &  & & \\
   & &    &\llap{$-($}-&6X^2&-&6X&-&6\rlap{$)$}& &   & & & & & &   & &  & & \\ \cline{4-9}
   & &    & &    & &9X&+&12\rlap{$\mathstrut=r$}& &   & & & & & &   & &  & & \\ \cline{7-9}
\end{array}
\]
Durch Nachrechnen kann man überprüfen, dass tatsächlich
\begin{align*}
bq
&=
X^4-X^3-7X^2-8X-6
\\
bq+r&=
X^4-X^3-7X^2+X+6 = a
\end{align*}
gilt.

Das Beispiel~\eqref{buch:polynome:eqn:divisionsaufgabe} war besonders
einfach, weil der führende Koeffizient des Divisorpolynomes $1$ war.
Für $b=2X^2+X+1$ funktioniert der Algorithmus dagegen nicht mehr.
Jedes für $q$ in Frage kommende Polynom vom Grad $2$ muss von der
Form $q=q_2X^2+q_1X+q_0$ sein.
Multipliziert man mit $b$, erhält man $bq=2q_2X^4 + (2q_1+q_2)X^3+\dots$.
Insbesondere ist es nicht möglich mit ganzzahligen Quotienten
$q_k\in\mathbb{Z}$ auch nur den ersten Koeffizienten von $a$ zu
erhalten.
Dazu müsste nämlich $a_n = 1 = 2q_2$ oder $q_2 = \frac12\not\in\mathbb{Z}$
sein.
Der Divisionsalgorithmus funktioniert also nur dann, wenn die 
Division durch den führenden Koeffizienten des Divisorpolynomes $b$ 
immer ausführbar ist.
Im Beispiel~\eqref{buch:polynome:eqn:divisionsaufgabe} war das der
Fall, weil der führende Koeffizient $1$ war.
Für beliebige Polynome $b\in R[X]$ ist dies aber nur dann immer der Fall,
wenn die Koeffizienten in Tat und Wahrheit einem Körper entstammen.

\begin{beispiel}
Im Folgenden betrachten wir daher nur noch Polynomringe mit Koeffizienten
in einem Körper $\Bbbk$.
In $\mathbb{Q}[X]$ ist die Division $a:b$ für die Polynome
\begin{equation}
\begin{aligned}
a(X) &= X^4 - X^3 -7X^2 + X + 6\\
b(X) &= 2X^2+X+1,
\end{aligned}
\label{buch:polynome:eqn:divisionsaufgabe2}
\end{equation}
problemlos durchführbar:
\[
\arraycolsep=1.4pt
\renewcommand{\arraystretch}{1.2}
\begin{array}{rcrcrcrcrcrcrcrcrcrcr}
X^4&-&       X^3&-&         7X^2&+&          X&+&           6&:&2X^2&+&X&+&1&=&\frac12X^2&-&\frac34X&-\frac{27}{8} = q\\
\llap{$-($}X^4&+&\frac12X^3&+&   \frac12X^2\rlap{$)$}& &           & &            & &    & & & & & &          & &        &             \\ \cline{1-5}
   &-&\frac32X^3&-&\frac{15}2X^2&+&          X& &            & &    & & & & & &          & &        &             \\
   &\llap{$-($}-&\frac32X^3&-&\frac{ 3}4X^2&-&\frac{ 3}4X\rlap{$)$}& &            & &    & & & & & &          & &        &             \\\cline{2-7}
   & &          &-&\frac{27}4X^2&+&\frac{ 7}4X&+&           6& &    & & & & & &          & &        &             \\
   & &          &\llap{$-($}-&\frac{27}4X^2&-&\frac{27}8X&-&\frac{27}{8}\rlap{$)$}& &    & & & & & &          & &        &             \\\cline{4-9}
   & &          & &             & &\frac{41}8X&+&\frac{75}{8}\rlap{$\mathstrut=r$}& &    & & & & & &          & &        &             \\
\end{array}
\]
Der Algorithmus funktioniert selbstverständlich genauso in $\mathbb{R}[X]$
oder $\mathbb{C}[X]$ und ebenso in den in
Kapitel~\ref{buch:chapter:endliche-koerper} studierten endlichen Körpern.
\end{beispiel}

\subsubsection{Euklidische Ringe und Faktorzerlegung}
Der Polynomring $\Bbbk[X]$ hat noch eine weitere Eigenschaft, die ihn
von einem gewöhnlichen Ring unterschiedet.
Der Polynomdivisionsalgorithmus findet zu zwei Polynomen $f,g\in\Bbbk[X]$
den Quotienten $q\in\Bbbk[X]$ und den Rest $r\in\Bbbk[X]$ mit
$f=qg+r$, wobei ausserdem $\deg r<\deg g$ ist.

\begin{definition}
Ein {\em euklidischer Ring} $R$ ist ein nullteilerfreier Ring mit einer
\index{euklischer Ring}%
Gradfunktion $\deg\colon R\setminus\{0\}\to\mathbb{N}$ mit folgenden
Eigenschaften
\begin{enumerate}
\item Für $x,y\in R$ gilt $\deg(xy) \ge \deg(x)$.
\item Für alle $x,y\in R$ gibt es $q,r\in R$ mit $x=qy+r$ mit
$\deg(y)>\deg(x)$
\label{buch:20-polynome:def:euklidischerring-2}
\end{enumerate}
Bedingung~\ref{buch:20-polynome:def:euklidischerring-2} ist die
{\em Division mit Rest}.
\index{Gradfunktion}%
\index{Division mit Rest}%
\index{euklidischer Ring}%
\end{definition}

Die ganzen Zahlen $\mathbb{Z}$ bilden einen euklidischen Ring mit der 
Gradfunktion $\deg(z)=|z|$ für $z\in \mathbb{Z}$.
Aus dem Divisionsalgorithmus für ganze Zahlen leiten sich alle grundlegenden
Eigenschaften über Teilbarkeit und Primzahlen ab.
Eine Zahl $x$ ist teilbar durch $y$, wenn $x=qy$ mit $q\in \mathbb{Z}$,
es gibt Zahlen $p\in\mathbb{Z}$, die keine Teiler haben und jede Zahl
kann auf eindeutige Art und Weise in ein Produkt von Primfaktoren
zerlegt werden.

\subsubsection{Irreduzible Polynome}
Das Konzept der Primzahl lässt sich wie folgt in die Welt der Polynomringe
übertragen.
\index{Primzahl}%

\begin{definition}
Ein Polynom $f\in R[X]$ heisst irreduzibel, wenn es keine Faktorisierung $f=gh$
in Faktoren $g,h\in R[X]$ mit $\deg(g)>0$ und $\deg(h) >0$ gibt.
\index{irreduzibles Polynom}%
\end{definition}

\begin{beispiel}
Polynome ersten Grades $aX+b$ sind immer irreduzibel, da sie bereits
minimalen Grad haben.

Sei jetzt $f=X^2+bX+c$ ein quadratisches Polynom in $\mathbb{Q}[X]$.
Wenn es faktorisierbar sein soll, dann müssen die Faktoren Polynome
ersten Grades sein, also $f=(X-x_1)(X-x_2)$ mit $x_i\in\mathbb{Q}$.
Die Zahlen $x_i$ die einzigen möglichen Lösungen für $x_i$ können mit
der Lösungsformel
\[
x_i = -\frac{b}2\pm\sqrt{\frac{b^2}{4}-c}
\]
für die quadratische Gleichung
gefunden werden.
Die Faktorisierung ist also genau dann möglich, wenn $b^2/4-c$ ein 
Quadrat in $\mathbb{Q}$ ist.
In $\mathbb{R}$ ist das Polynom faktorisierbar, wenn $b^2-4c\ge 0$ ist.
In $\mathbb{C}$ gibt es keine Einschränkung, die Wurzel zu ziehen,
in $\mathbb{C}$ gibt es also keine irreduziblen Polynome vom Grad $2$.
\end{beispiel}

\subsubsection{Faktorisierung in einem Polynomring}
Ein Polynomring ist ganz offensichtlich auch ein euklidischer Ring.
Wir erwarten daher die entsprechenden Eigenschaften auch in einem
Polynomring.
Allerdings ist eine Faktorzerlegung nicht ganz eindeutig.
Wenn das Polynom $f\in\mathbb{Z}[X]$ die Faktorisierung
$f=g\cdot h$ mit $g,h\mathbb{Z}[X]$ hat, dann
ist $rg\cdot r^{-1}h$ ebenfalls eine Faktorisierung für jedes $r =\pm1$.
Dasselbe gilt in $\mathbb{Q}$ für jedes $r\in \mathbb{Q}^*$.
Faktorisierung ist also nur eindeutig bis auf Elemente der
Einheitengruppe des Koeffizientenringes.
Diese Mehrdeutigkeit kann in den Polynomringen $\Bbbk[X]$ 
überwunden werden, indem die Polynome normiert werden.

\begin{satz}
Ein normiertes Polynom $f\in \Bbbk[X]$ kann in
normierte Faktoren $g_1,\dots,g_k\in\Bbbk[X]$ zerlegt werden, so dass
$f=g_1\cdot\ldots\cdot g_k$, wobei die Faktoren irreduzibel sind.
Zwei solche Faktorisierungen unterscheiden sich nur durch die Reihenfolge
der Faktoren.
Ein Polynom $f\in \Bbbk[X]$ kann in ein Produkt $a_n g_1\cdot\ldots\cdot g_k$
zerlegt werden, wobei die normierten Faktoren $g_i$ bis auf die Reihenfolge
eindeutig sind.
\end{satz}



%
% vektoren.tex -- Darstellung von Polynomen als Vektoren
%
% (c) 2021 Prof Dr Andreas Müller, OST Ostschweizer Fachhochschule Rapperswil
%
\section{Polynome als Vektoren
\label{buch:section:polynome:vektoren}}
\rhead{Polynome als Vektoren}
Ein Polynom
\[
p(X) = a_nX^n + a_{n-1}X^{n-1} + \dots a_1X+a_0
\]
mit Koeffizienten in einem Ring $R$
ist spezifiziert, wenn die Koeffizienten $a_k$ bekannt sind.
Die Potenzen von $X$ dienen hier nur dazu, die verschiedenen
Koeffizienten zu unterscheiden.
Das Polynom $p(X)$ vom Grad $n$ ist also auch gegeben durch den
$n+1$-dimensionalen Vektor
\[
\begin{pmatrix}
a_0\\
a_1\\
\vdots\\
a_{n-1}\\
a_{n}
\end{pmatrix}
\in
R^{n+1}.
\]
Diese Darstellung eines Polynoms gibt auch die Addition von Polynomen
und die Multiplikation von Polynomen mit Skalaren aus $R$ korrekt wieder.
Die Abbildung
\[
\varphi
\colon  R^{n+1} \to R[X]
:
\begin{pmatrix}a_0\\\vdots\\a_n\end{pmatrix}
\mapsto
a_nX^n + a_{n-1}X^{n-1}+\dots+a_1X+a_0
\]
von Vektoren auf Polynome
erfüllt also
\[
\varphi( \lambda a) = \lambda \varphi(a)
\qquad\text{und}\qquad
\varphi(a+b) = \varphi(a) + \varphi(b)
\]
und ist damit eine lineare Abbildung.
Umgekehrt kann man auch zu jedem Polynom $p(X)$ vom Grad~$\le n$ einen
Vektor finden, der von $\varphi$ auf das Polynom $p(X)$ abgebildet wird.
Die Abbildung $\varphi$ ist also ein Isomorphismus
\[
\varphi
\colon
\{p\in R[X] \mid \deg(p) \le n\}
\overset{\cong}{\to}
R^{n+1}
\]
zwischen der Menge
der Polynome vom Grad $\le n$ auf $R^{n+1}$.
Für alle Rechnungen, bei denen es nur um Addition von Polynomen oder
um Multiplikation mit Skalaren geht, ist also diese vektorielle Darstellung
mit Hilfe von $\varphi$ eine zweckmässige Darstellung.

In zwei Bereichen ist die Beschreibung von Polynomen mit Vektoren allerdings
ungenügend: einerseits können Polynome beliebig hohen Grad haben,
während Vektoren in $R^{n+1}$ höchstens $n+1$ Komponenten haben können.
Andererseits geht bei der vektoriellen Beschreibung die multiplikative
Struktur vollständig verloren.

\subsection{Polynome beliebigen Grades
\label{buch:subsection:polynome:beliebigergrad}}
Ein Polynom
\[
q(X)
=
b_mX^m + b_{m-1}X^{m-1} + \dots + b_1X + b_0
\]
vom Grad $m<n$ kann dargestellt werden als ein Vektor
\[
\begin{pmatrix}
b_0\\
b_1\\
\vdots\\
b_{m-1}\\
b_{m}\\
0\\
\vdots
\end{pmatrix}
\in
R^{n+1}
\]
mit der Eigenschaft, dass die Komponenten mit Indizes
$m+1,\dots n$ verschwinden.
Polynome vom Grad $m<n$ bilden einen Unterraum der Polynome vom Grad $n$.
Wir können auch die $m+1$-dimensionalen Vektoren in den $n+1$-dimensionalen
Vektoren einbetten, indem wir die Vektoren durch ``Auffüllen'' mit Nullen
auf die richtige Länge bringen.
Es gibt also eine lineare Abbildung
\[
R^{m+1} \to R^{n+1}
\colon
\begin{pmatrix}
b_0\\b_1\\\vdots\\b_m
\end{pmatrix}
\mapsto
\begin{pmatrix}
b_0\\b_1\\\vdots\\b_m\\0\\\vdots
\end{pmatrix}
.
\]
Die Vektormengen $R^{k+1}$ sind also alle ineinandergeschachtelt, können aber
alle auf konsistente Weise mit der Abbildung $\varphi$ in den Polynomring
$R[X]$ abgebildet werden.
\begin{center}
\begin{tikzcd}[>=latex]
R \ar[r] \arrow[d, "\varphi"]
	&R^2 \ar[r] \arrow[d, "\varphi"]
		&R^3 \ar[r] \arrow[d, "\varphi"]
			&\dots \ar[r]
				&R^k \ar[r] \arrow[d, "\varphi"]
					&R^{k+1} \ar[r] \arrow[d, "\varphi"]
						&\dots
\\
R^{(0)}[X]\arrow[r,hook] \arrow[drrr,hook]
	&R^{(1)}[X]\arrow[r,hook] \arrow[drr,hook]
		&R^{(2)}[X]\arrow[r,hook] \arrow[dr,hook]
			&\dots\arrow[r,hook]
				&R^{(k-1)}[X]\arrow[r,hook] \arrow[dl,hook]
					&R^{(k)}[X]\arrow[r,hook] \arrow[dll,hook]
						&\dots
\\
	&
		&
			&R[X]
				&
					&
						&
\end{tikzcd}
\end{center}
In diesem Sinne können wir $R^m$ für $m<n$ als Teilmenge von $R^n$ betrachten
und $R^\infty$ als deren Vereinigung definieren.
Polynome in $R[X]$ sind also Vektoren beliebiger Länge mit Kompoenten
in $R$.

\subsection{Multiplikative Struktur
\label{buch:subsection:polynome:multiplikativestruktur}}
Den Polynomring $R[X]$ aus den Vektoren $R^{k}$ aufzubauen, bedeutet,
dass wir die multiplikative Struktur ignorieren.
Augrund der Rechenregeln für das Symbol $X$ können wir $X$ als einen
Multiplikationsoperator 
\[
{X\cdot} 
\colon R^{m} \to R^{n}
:
\begin{pmatrix}a_0\\a_1\\a_2\\\vdots\end{pmatrix}
\mapsto
\begin{pmatrix}0\\a_0\\a_1\\\vdots\end{pmatrix}
\]
betrachten.
Diese Operatoren setzen sich zusammen zu einem Operator
\[
{X\cdot} : R^\infty \to \infty,
\]
der die Multiplikation mit $X$ beschreibt.

Ist $p(X)$ ein Polynom, dann lässt sich die Multiplikation
von $p(X)$ mit Polynomen in $R[X]$ ebenfalls als Operator schreiben.
Die Potenz $X^k$ wirkt durch $k$-fache Iteration des Operators
$X\cdot$.
Das Polynom $p(X)$ wirkt als Linearkombination der Operatoren $(X\cdot)^k$,
entspricht also dem Operator, den man durch Einsetzen von $X\cdot$
in das Polynom erhalten kann:
\[
p(X\cdot)
=
a_n(X\cdot)^n + a_{n-1}(X\cdot)^{n+1} + \dots + a_1(X\cdot) + a_0
:
R^\infty \to R^\infty
:
q(X) 
\mapsto
p(X)q(X).
\]
Man kann den Operator $X\cdot$ oder den iterierten Operator
$(X\cdot)^k$ auch in Matrixform darstellen:
\begin{align*}
{X\cdot}
&=
\begin{pmatrix}
0&0&0&0&\dots\\
1&0&0&0&\dots\\
0&1&0&0&\dots\\
0&0&1&0&\dots\\
\vdots&\vdots&\vdots&\ddots&\ddots
\end{pmatrix},
&
(X\cdot)^k
&=
\begin{pmatrix}
  0   &  0   &  0   &  0   &\dots\\
\vdots&\vdots&\vdots&\vdots&     \\
  0   &  0   &  0   &  0   &\dots\\
  1   &  0   &  0   &  0   &\dots\\
  0   &  1   &  0   &  0   &\dots\\
  0   &  0   &  1   &  0   &\dots\\
\vdots&\vdots&\vdots&\ddots&\ddots
\end{pmatrix}.
\end{align*}
In der Matrix für $(X\cdot)^k$ steht die erste $1$ auf der
$k+1$-ten Zeile.
Der zum Polynom $p(X)$ gehörige Operator $p(X\cdot)$ bekommt
damit die Matrix
\[
p(X\cdot)
=
\begin{pmatrix}
a_0    & 0     &  0   &  0   &  0   & \dots  \\
a_1    &a_0    &  0   &  0   &  0   & \dots  \\
a_2    &a_1    & a_0  &  0   &  0   & \dots  \\
a_3    &a_2    & a_1  & a_0  &  0   & \dots  \\
a_4    &a_3    & a_2  & a_1  & a_0  & \dots  \\
\vdots &\vdots &\vdots&\vdots&\vdots&\ddots
\end{pmatrix}.
\]
Da die Matrix-Operation als Produkt
$\text{Zeile}\times\text{Spalte}$ ausgeführt wird,
kann man erkennen, dass das Polynomprodukt auch auf
eine Faltung hinausläuft:
Die Multiplikation einer Zeile der Matrix $p(X\cdot)$  mit
einem Spaltenvektor $b$ multipliziert den gespiegelten und verschobenen
Vektor der Koeffizienten $a$ mit den Koeffizienten $b$.

Die wichtigste Lehre aus obigen Ausführungen aber ist
die Beobachtung, dass sich eine ganz allgemeine Algebra
wie die der Polynome auf sehr direkte Art und Weise 
abbilden lässt in eine Algebra von Matrizen auf einem
geeigneten Vektorraum.
Im vorliegenden Fall sind das zwar ``unendliche''
Matrizen, in zukünftigen Beispielen werden wir das
selbe Prinzip jedoch in Aktion sehen in Situationen,
wo eine Operation auf einem endlichen Vektorraum
und ``gewöhnliche'' Matrizen entstehen.
Die Möglichkeit, beliebige Polynome solcher Operatoren
zu berechnen, erlaubt uns, mehr über den Operator 
herauszufinden

Dies eröffnet vielfältige Möglichkeiten, auf einfachere
Art mit den Operatoren zu rechnen.
In Kapitel~\ref{buch:chapter:eigenwerte-und-eigenvektoren}
wird sich daraus eine Reihe von Normalformen einer Matrix
ergeben sowie die Möglichkeit, für viele Matrizen $A$
die Matrix $f(A)$ für eine grosse Zahl von praktisch
interessanten Funktionen $f(z)$ zu berechnen.


%%
% permutationsmatrizen.tex -- Permutationsmatrizen
%
% (c) 2020 Prof Dr Andreas Müller, Hochschule Rapperswil
%
\section{Permutationsmatrizen
\label{buch:section:permutationsmatrizen}}
\rhead{Permutationsmatrizen}
Die Eigenschaft, dass eine Vertauschung das Vorzeichen kehrt, ist
eine wohlbekannte Eigenschaft der Determinanten.
In diesem Abschnitt soll daher eine Darstellung von Permutationen
als Matrizen vorgestellt werden und die Verbindung zwischen dem
Vorzeichen einer Permutation und der Determinanten hergestellt
werden.
\index{Determinante}%

\subsection{Matrizen}
Gegeben sei jetzt eine Permutation $\sigma\in S_n$. 
Aus $\sigma$ lässt sich eine lineare Abbildung $\Bbbk^n\to\Bbbk^n$
konstruieren, die die Standardbasisvektoren permutiert, also
\[
f_{\sigma}\colon
\Bbbk^n \to \Bbbk^n
:
\left\{
\begin{aligned}
e_1&\mapsto e_{\sigma(1)} \\
e_2&\mapsto e_{\sigma(2)} \\
\vdots&\\
e_n&\mapsto e_{\sigma(n)}
\end{aligned}
\right.
\]
Die Matrix $P_\sigma$ der linearen Abbildung $f_{\sigma}$ hat in Spalte $i$
genau eine $1$ in der Zeile $\sigma(i)$, also
\[
(P_\sigma)_{i\!j} = \delta_{j\sigma(i)}.
\]

\begin{beispiel}
Die zur Permutation
\[
\begin{pmatrix}
1&2&3&4&5&6\\
2&1&3&5&6&4
\end{pmatrix}
\]
gehörige lineare Abbildung $f_\sigma$ hat die Matrix
\[
A_\sigma
=
\begin{pmatrix}
0&1&0&0&0&0\\
1&0&0&0&0&0\\
0&0&1&0&0&0\\
0&0&0&0&0&1\\
0&0&0&1&0&0\\
0&0&0&0&1&0
\end{pmatrix}
\qedhere
\]
\end{beispiel}

\begin{definition}
\label{buch:permutationen:def:permutationsmatrix}
\index{Permutationsmatrix}%
Eine {\em Permutationsmatrix} ist eine Matrix $P\in M_n(\Bbbk)$,
die in jeder Zeile und Spalte genau eine $1$ enthält,
während alle anderen Matrixelemente $0$ sind.
\end{definition}

Es ist klar, dass aus einer Permutationsmatrix auch die Permutation
der Standardbasisvektoren abgelesen werden kann.
\index{Standardbasisvektor}%
Die Verknüpfung von Permutationen wird zur Matrixmultiplikation
\index{Matrixmultiplikation}%
von Permutationsmatrizen, die Zuordnung $\sigma\mapsto P_\sigma$
ist also ein Homomorphismus
\index{Homomorphismus}%
$S_n \to M_n(\Bbbk^n)$,
es ist
$P_{\sigma_1\sigma_2}=P_{\sigma_1}P_{\sigma_2}$.
$\sigma$ heisst gemäss Definition~\ref{buch:vektorenmatrizen:def:darstellung}
auch Darstellung der Gruppe $S_n$.
\index{Darstellung}%

\subsection{Transpositionen}
Transpositionen sind Permutationen, die genau zwei Elemente von $[n]$
vertauschen.
Wir ermitteln jetzt die Permutationsmatrix der Transposition $\tau=\tau_{i\!j}$.
Sie ist
\[
P_{\tau_{i\!j}}
=
\begin{pmatrix}
1&      & &      &     &      & &      & \\
 &\ddots& &      &     &      & &      & \\
 &      &1&      &     &      & &      & \\
 &      & &0     &\dots&1     & &      & \\
 &      & &\vdots&     &\vdots& &      & \\
 &      & &1     &\dots&0     & &      & \\
 &      & &      &     &      &1&      & \\
 &      & &      &     &      & &\ddots& \\
 &      & &      &     &      & &      &1
\end{pmatrix}.
\]

Die Permutation $\sigma$ mit dem Zyklus $1\to 2\to\dots\to l-1\to l\to 1$
der Länge $l$ kann aus aufeinanderfolgenden Transpositionen zusammengesetzt
werden, die zugehörigen Permutationsmatrizen sind
\begin{align*}
P_\sigma
&=
P_{\tau_{12}}
P_{\tau_{23}}
P_{\tau_{34}}\dots
P_{\tau_{l-2,l-1}}
P_{\tau_{l-1,l}}
\\
&=
\begin{pmatrix}
0&1&0&0&\dots\\
1&0&0&0&\dots\\
0&0&1&0&\dots\\
0&0&0&1&\dots\\
\vdots&\vdots&\vdots&\vdots&\ddots
\end{pmatrix}
\begin{pmatrix}
1&0&0&0&\dots\\
0&0&1&0&\dots\\
0&1&0&0&\dots\\
0&0&0&1&\dots\\
\vdots&\vdots&\vdots&\vdots&\ddots
\end{pmatrix}
\begin{pmatrix}
1&0&0&0&\dots\\
0&1&0&0&\dots\\
0&0&0&1&\dots\\
0&0&1&0&\dots\\
\vdots&\vdots&\vdots&\vdots&\ddots
\end{pmatrix}
\dots
\\
&=
\begin{pmatrix}
0&0&1&0&\dots\\
1&0&0&0&\dots\\
0&1&0&0&\dots\\
0&0&0&1&\dots\\
\vdots&\vdots&\vdots&\vdots&\ddots
\end{pmatrix}
\begin{pmatrix}
1&0&0&0&\dots\\
0&1&0&0&\dots\\
0&0&0&1&\dots\\
0&0&1&0&\dots\\
\vdots&\vdots&\vdots&\vdots&\ddots
\end{pmatrix}
\cdots
\\
&=
\begin{pmatrix}
0&0&0&1&\dots\\
1&0&0&0&\dots\\
0&1&0&0&\dots\\
0&0&1&0&\dots\\
\vdots&\vdots&\vdots&\vdots&\ddots
\end{pmatrix}
\\
&\quad\vdots\\
&=
\begin{pmatrix}
0&0&0&0&\dots&0&1\\
1&0&0&0&\dots&0&0\\
0&1&0&0&\dots&0&0\\
0&0&1&0&\dots&0&0\\
\vdots&\vdots&\vdots&\vdots&\ddots&\vdots&\vdots\\
0&0&0&0&\dots&1&0
\end{pmatrix}.
\end{align*}

\subsection{Determinante und Vorzeichen}
Die Transpositionen haben Permutationsmatrizen, die aus der Einheitsmatrix
\index{Einheitsmatrix}%
\index{Determinante}%
entstehen, indem genau zwei Zeilen vertauscht werden.
Die Determinante einer solchen Permutationsmatrix ist
\[
\det P_{\tau} = - \det I = -1 = \operatorname{sgn}(\tau).
\]
Nach der Produktregel für die Determinante folgt für eine Darstellung
der Permutation $\sigma=\tau_1\dots\tau_l$ als Produkt von Transpositionen,
dass
\begin{equation}
\det P_{\sigma}
=
\det P_{\tau_1} \cdots \det P_{\tau_l}
=
(-1)^l
=
\operatorname{sgn}(\sigma).
\label{buch:permutationen:determinante}
\end{equation}
Das Vorzeichen einer Permutation ist also identisch mit der Determinante
der zugehörigen Permutationsmatrix.



%\input{chapters/20-polynome/minimalpolynom.tex}


