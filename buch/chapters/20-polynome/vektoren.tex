%
% vektoren.tex -- Darstellung von Polynomen als Vektoren
%
% (c) 2021 Prof Dr Andreas Müller, OST Ostschweizer Fachhochschule Rapperswil
%
\section{Polynome als Vektoren
\label{buch:section:polynome:vektoren}}
\rhead{Polynome als Vektoren}
Ein Polynom
\[
p(X) = a_nX^n + a_{n-1}X^{n-1} + \dots a_1X+a_0
\]
mit Koeffizienten in einem Ring $R$
ist spezifiziert, wenn die Koeffizienten $a_k$ bekannt sind.
Die Potenzen von $X$ dienen hier nur dazu, die verschiedenen
Koeffizienten zu unterscheiden.
Das Polynom $p(X)$ vom Grad $n$ ist also auch gegeben durch den
$n+1$-dimensionalen Vektor
\[
\begin{pmatrix}
a_0\\
a_1\\
\vdots\\
a_{n-1}\\
a_{n}
\end{pmatrix}
\in
R^{n+1}.
\]
Diese Darstellung eines Polynoms gibt auch die Addition von Polynomen
und die Multiplikation von Polynomen mit Skalaren aus $R$ korrekt wieder.
Die Abbildung von Vektoren auf Polynome
\[
\varphi
\colon  R^{n+1} \to R[X]
:
\begin{pmatrix}a_0\\\vdots\\a_n\end{pmatrix}
\mapsto
a_nX^n + a_{n-1}X^{n-1}+\dots+a_1X+a_0
\]
erfüllt also
\[
\varphi( \lambda a) = \lambda \varphi(a)
\qquad\text{und}\qquad
\varphi(a+b) = \varphi(a) + \varphi(b)
\]
und ist damit eine lineare Abbildung.
Umgekehrt kann man auch zu jedem Polynom $p(X)$ vom Grad~$\le n$ einen
Vektor finden, der von $\varphi$ auf das Polynom $p(X)$ abgebildet wird.
Die Abbildung $\varphi$ ist also ein Isomorphismus
\[
\varphi
\colon
\{p\in R[X] \mid \deg(p) \le n\}
\overset{\cong}{\to}
R^{n+1}
\]
zwischen der Menge
der Polynome vom Grad $\le n$ auf $R^{n+1}$.
Für alle Rechnungen, bei denen es nur um Addition von Polynomen oder
um Multiplikation mit Skalaren geht, ist also diese vektorielle Darstellung
mit Hilfe von $\varphi$ eine zweckmässige Darstellung.

In zwei Bereichen ist die Beschreibung von Polynomen mit Vektoren allerdings
ungenügend: einerseits können Polynome können beliebig hohen Grad haben,
während Vektoren in $R^{n+1}$ höchstens $n+1$ Komponenten haben können.
Andererseits geht bei der vektoriellen Beschreibung die multiplikative
Struktur vollständig verloren.

\subsection{Polynome beliebigen Grades
\label{buch:subsection:polynome:beliebigergrad}}
Ein Polynom
\[
q(X)
=
b_mX^m + b_{m-1}X^{m-1} + \dots + b_1X + b_0
\]
vom Grad $m<n$ kann dargestellt werden als ein Vektor
\[
\begin{pmatrix}
b_0\\
b_1\\
\vdots\\
b_{m-1}\\
b_{m}\\
0\\
\vdots
\end{pmatrix}
\in
R^{n+1}
\]
mit der Eigenschaft, dass die Komponenten mit Indizes
$m+1,\dots n$ verschwinden.
Polynome vom Grad $m<n$ bilden einen Unterraum der Polynome vom Grad $n$.
Wir können auch die $m+1$-dimensionalen Vektoren in den $n+1$-dimensionalen
Vektoren einbetten, indem wir die Vektoren durch ``Auffüllen'' mit Nullen
auf die richtige Länge bringen.
Es gibt also eine lineare Abbildung
\[
R^{m+1} \to R^{n+1}
\colon
\begin{pmatrix}
b_0\\b_1\\\vdots\\b_m
\end{pmatrix}
\mapsto
\begin{pmatrix}
b_0\\b_1\\\vdots\\b_m\\0\\\vdots
\end{pmatrix}
.
\]
Die Vektormengen $R^{k+1}$ sind also alle ineinandergeschachtelt, können aber
alle auf konsistente Weise mit der Abbildung $\varphi$ in den Polynomring
$R[X]$ abgebildet werden.
\begin{center}
\begin{tikzcd}[>=latex]
R \ar[r] \arrow[d, "\varphi"]
	&R^2 \ar[r] \arrow[d, "\varphi"]
		&R^3 \ar[r] \arrow[d, "\varphi"]
			&\dots \ar[r]
				&R^k \ar[r] \arrow[d, "\varphi"]
					&R^{k+1} \ar[r] \arrow[d, "\varphi"]
						&\dots
\\
R^{(0)}[X]\arrow[r,hook] \arrow[drrr,hook]
	&R^{(1)}[X]\arrow[r,hook] \arrow[drr,hook]
		&R^{(2)}[X]\arrow[r,hook] \arrow[dr,hook]
			&\dots\arrow[r,hook]
				&R^{(k-1)}[X]\arrow[r,hook] \arrow[dl,hook]
					&R^{(k)}[X]\arrow[r,hook] \arrow[dll,hook]
						&\dots
\\
	&
		&
			&R[X]
				&
					&
						&
\end{tikzcd}
\end{center}
In diesem Sinne können wir $R^m$ für $m<n$ als Teilmenge von $R^n$ betrachten
und $R^\infty$ als deren Vereinigung definieren.
Polynome in $R[X]$ sind also Vektoren beliebiger Länge mit Kompoenten
in $R$.

\subsection{Multiplikative Struktur
\label{buch:subsection:polynome:multiplikativestruktur}}
Den Polynomring $R[X]$ aus den Vektoren $R^{k}$ aufzubauen, bedeutet,
dass wir die multiplikative Struktur ignorieren.
Augrund der Rechenregeln für das Symbol $X$ können wir $X$ als einen
Multiplikationsoperator 
\[
{X\cdot} 
\colon R^{m} \to R^{n}
:
\begin{pmatrix}a_0\\a_1\\a_2\\\vdots\end{pmatrix}
\mapsto
\begin{pmatrix}0\\a_0\\a_1\\\vdots\end{pmatrix}
\]
betrachten.
Diese Operatoren setzen sich zusammen zu einem Operator
\[
{X\cdot} \colon R^\infty \to \infty,
\]
der die Multiplikation mit $X$ beschreibt.

Ist $p(X)$ ein Polynom, dann lässt sich die Multiplikation
in von Polynome mit $R[X]$ ebenfalls als Operator schreiben.
Die Potenz $X^k$ wirkt durch $k$-fache Iteration des Operators
$X\cdot$.
Das Polynom $p(X)$ wirkt als Linearkombination der Operatoren $(X\cdot)^k$,
entspricht also dem Operator, den man durch Einsetzen von $X\cdot$
in das Polynom erhalten kann:
\[
p(X\cdot)
=
a_n(X\cdot)^n + a_{n-1}(X\cdot)^{n+1} + \dots + a_1(X\cdot) + a_0
\colon
R^\infty \to R^\infty
:
q(X) 
\mapsto
p(X)q(X).
\]
Man kann den Operator $X\cdot$ oder den iterierten Operator
$(X\cdot)^k$ auch in Matrixform darstellen:
\begin{align*}
{X\cdot}
&=
\begin{pmatrix}
0&0&0&0&\dots\\
1&0&0&0&\dots\\
0&1&0&0&\dots\\
0&0&1&0&\dots\\
\vdots&\vdots&\vdots&\ddots&\ddots
\end{pmatrix},
&
(X\cdot)^k
&=
\begin{pmatrix}
  0   &  0   &  0   &  0   &\dots\\
\vdots&\vdots&\vdots&\vdots&     \\
  0   &  0   &  0   &  0   &\dots\\
  1   &  0   &  0   &  0   &\dots\\
  0   &  1   &  0   &  0   &\dots\\
  0   &  0   &  1   &  0   &\dots\\
\vdots&\vdots&\vdots&\ddots&\ddots
\end{pmatrix}.
\end{align*}
In der Matrix für $(X\cdot)^k$ steht die erste $1$ auf der
$k+1$-ten Zeile.
Der zum Polynom $p(X)$ gehörige Operator $p(X\cdot)$ bekommt
damit die Matrix
\[
p(X\cdot)
=
\begin{pmatrix}
a_0    & 0     &  0   &  0   &  0   & \dots  \\
a_1    &a_0    &  0   &  0   &  0   & \dots  \\
a_2    &a_1    & a_0  &  0   &  0   & \dots  \\
a_3    &a_2    & a_1  & a_0  &  0   & \dots  \\
a_4    &a_3    & a_2  & a_1  & a_0  & \dots  \\
\vdots &\vdots &\vdots&\vdots&\vdots&\ddots
\end{pmatrix}.
\]
Da die Matrix-Operation als Produkt
$\text{Zeile}\times\text{Spalte}$ ausgeführt wird,
kann man erkennen, dass das Polynomprodukt auch auf
eine Faltung hinausläuft:
Die Multiplikation einer Zeile der Matrix $p(X\cdot)$  mit
einem Spaltenvektor $b$ multipliziert den gespiegelten und verschobenen
Vektor der Koeffizienten $a$ mit den Koeffizienten $b$.

Die wichtigste Lehre aus obigen Ausführungen aber ist
die Beobachtung, dass sich eine ganz allgemeine Algebra
wie die der Polynome auf sehr direkte Art und Weise 
abbilden lässt in eine Algebra von Matrizen auf einem
geeigneten Vektorraum.
Im vorliegenden Fall sind das zwar ``undendliche''
Matrizen, in zukünftigen Beispielen werden wir das
selbe Prinzip jedoch in Aktion sehen in Situationen,
wo eine Operation auf einem endlichen Vektorraum
und ``gewöhnliche'' Matrizen entstehen.
Die Möglichkeit, beliebige Polynome solcher Operatoren
zu berechnen, erlaubt uns, mehr über den Operator 
herauszufinden

Dies eröffnet vielfältige Möglichkeiten, auf einfachere
Art mit den Operatoren zu rechnen.
In Kapitel~\ref{buch:chapter:eigenwerte-und-eigenvektoren}
wird sich daraus eine Reihe von Normalformen einer Matrix
ergeben sowie die Möglichkeit, für viele Matrizen $A$
die Matrix $f(A)$ für eine grosse Zahl von praktisch
interessanten Funktionen $f(z)$ zu berechnen.

