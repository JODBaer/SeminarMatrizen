Eine einem Glücksspiel gewinnt oder verliert ein Spieler in
jeder Runde mit Wahrscheinlichkeit $p$ bzw.~$1-p$ eine Einheit, bis
sein Anfangskapital $b$ entweder aufgebraucht ist oder verdoppelt
worden ist.
Wie lange dauert es im Mittel, bis einer dieser Fälle eintritt für
$b=5$ und $p=\frac12$?

\begin{loesung}
Die Spielregeln definieren eine Markov-Kette mit Zuständen
$\mathcal{S}=\{0,\dots,2b\}$ und Übergangsmatrix
\[
T=\left(
\begin{array}{c|ccccc|c}
   1   &  1-p  &   0   &\dots  &   0   &   0   &   0   \\
\hline
   0   &   0   &  1-p  &\dots  &   0   &   0   &   0   \\
   0   &   p   &   0   &\dots  &   0   &   0   &   0   \\
   0   &   0   &   p   &\dots  &   0   &   0   &   0   \\
   0   &   0   &   0   &\dots  &   0   &   0   &   0   \\
\vdots &\vdots &\vdots &       &\vdots &\vdots &\vdots \\
   0   &   0   &   0   &\dots  &   0   &  1-p  &   0   \\
   0   &   0   &   0   &\dots  &   p   &   0   &   0   \\
\hline
   0   &   0   &   0   &\dots  &   0   &   p   &   1   
\end{array}
\right).
\]
Offenbar sind die beiden Zustände $0$ und $2b$ absorbierend, alle
anderen transient.

Sortieren wir die Zustände um, so dass zuerst die absorbierenden
Zustände gelistet werden, dann bekommt die Übergangsmatrix die Form
\[
T=\left(
\begin{array}{cc|ccccc}
   1   &   0   &  1-p  &   0   &\dots  &   0   &   0   \\
   0   &   1   &   0   &   0   &\dots  &   0   &   p   \\
\hline
   0   &   0   &   0   &  1-p  &\dots  &   0   &   0   \\
   0   &   0   &   p   &   0   &\dots  &   0   &   0   \\
   0   &   0   &   0   &   p   &\dots  &   0   &   0   \\
   0   &   0   &   0   &   0   &\dots  &   0   &   0   \\
\vdots &\vdots &\vdots &\vdots &\ddots &\vdots &\vdots \\
   0   &   0   &   0   &   0   &\dots  &   0   &  1-p  \\
   0   &   0   &   0   &   0   &\dots  &   p   &   0   
\end{array}
\right).
\]
Die Matrizen $R$ und $Q$ sind daher
\begin{align*}
R&=\begin{pmatrix}
  1-p  &   0   &\dots  &   0   &   0   \\
   0   &   0   &\dots  &   0   &   p   
\end{pmatrix}
\\
Q&=\begin{pmatrix}
   0   &  1-p  &\dots  &   0   &   0   \\
   p   &   0   &\dots  &   0   &   0   \\
   0   &   p   &\dots  &   0   &   0   \\
   0   &   0   &\dots  &   0   &   0   \\
\vdots &\vdots &\ddots &\vdots &\vdots \\
   0   &   0   &\dots  &   0   &  1-p  \\
   0   &   0   &\dots  &   p   &   0   
\end{pmatrix}.
\end{align*}
Die Fundamentalmatrix $N$ ist die Summe der geometrischen Reihe
\[
N = I + Q + Q^2 + \dots = \sum_{k=0}^\infty Q^k = (I-Q)^{-1}.
\]
Mit einem Computeralgebra-Programm wie Maxima kann man die Fundamentalmatrix
berechnen kann.
Für $b=5$ und $p=\frac12$ findet man die Matrix
\[
N=\begin{pmatrix}
\frac{11}{6}&\frac{10}{6}&\frac{9}{6}&\frac{8}{6}&\frac{7}{6}&\frac{6}{6}&\frac{5}{6}&\frac{4}{6}&\frac{3}{6}&\frac{2}{6}&\frac{1}{6}\\
\frac{5}{3} &\frac{10}{3}&\frac{9}{3}&\frac{8}{3}&\frac{7}{3}&\frac{6}{3}&\frac{5}{3}&\frac{4}{3}&\frac{3}{3}&\frac{2}{3}&\frac{1}{3}\\
\frac{3}{2} & \frac{6}{2} & \frac{9}{2} & \frac{8}{2} & \frac{7}{2} & \frac{6}{2} & \frac{5}{2} & \frac{4}{2} & \frac{3}{2} & \frac{2}{2} & \frac{1}{2} \\
\frac{4}{3}&\frac{8}{3}&\frac{12}{3}& \frac{16}{3}& \frac{14}{3}& \frac{12}{3}& \frac{10}{3}& \frac{8}{3}& \frac{6}{3}& \frac{4}{3}& \frac{2}{3}\\
\frac{7}{6}& \frac{14}{6}& \frac{21}{6}& \frac{28}{6}& \frac{35}{6}& \frac{30}{6}& \frac{25}{6}& \frac{20}{6}& \frac{15}{6}& \frac{10}{6}& \frac{5}{6}\\
1&2&3&4&5&6&5&4&3&2&1\\
\frac{5}{6}& \frac{10}{6}& \frac{15}{6}& \frac{20}{6}& \frac{25}{6}& \frac{30}{6}& \frac{35}{6}& \frac{28}{6}& \frac{21}{6}& \frac{14}{6}& \frac{7}{6}\\
\frac{2}{3}& \frac{4}{3}& \frac{6}{3}& \frac{8}{3}& \frac{10}{3}& \frac{12}{3}& \frac{14}{3}& \frac{16}{3}& \frac{12}{3}& \frac{8}{3}& \frac{4}{3}\\
\frac{1}{2}& \frac{2}{2}& \frac{3}{2}& \frac{4}{2}& \frac{5}{2}& \frac{6}{2}& \frac{7}{2}& \frac{8}{2}& \frac{9}{2}& \frac{6}{2}& \frac{3}{2}\\
\frac{1}{3}& \frac{2}{3}& \frac{3}{3}& \frac{4}{3}& \frac{5}{3}& \frac{6}{3}& \frac{7}{3}& \frac{8}{3}& \frac{9}{3}& \frac{10}{3}& \frac{5}{3}\\
\frac{1}{6}& \frac{2}{6}& \frac{3}{6}& \frac{4}{6}& \frac{5}{6}& \frac{6}{6}& \frac{7}{6}& \frac{8}{6}& \frac{9}{6}& \frac{10}{6}& \frac{11}{6}
\end{pmatrix}.
\]
Da in diesem Fall $Q$ symmetrisch ist, ist auch $N$ symmetrisch.

Nach Satz~\ref{buch:markov:satz:absorptionszeit}
liest man aus der Fundamentalmatrix ab, dass
für ein Startkapital von $b=5$ und $p=\frac12$
die erwarte Absorbtionszeit $E(K)$ die Summe der Einträge in der mittleren Spalte
der Matrix sind, also $E(K)=1+2+3+4+5+6+5+\dots+1=36$ ist.
\end{loesung}

Die Aufgabe kann auch allgemeiner für $p\ne\frac12$ gelöst werden, da
man für die Fundamentalmatrix explizite Formeln finden kann \cite{skript:rudolph}.

