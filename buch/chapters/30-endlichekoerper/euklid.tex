%
% euklid.tex
%
% (c) 2019 Prof Dr Andreas Müller, Hochschule Rapperswil
%
\section{Der euklidische Algorithmus
\label{buch:section:euklid}}
\rhead{Der euklidische Algorithmus}
Der euklidische Algorithmus bestimmt zu zwei gegebenen ganzen
Zahlen $a$ und $b$ den grössten gemeinsamen Teiler $g$.
Wir schreiben $g|a$ für ``$g$ ist Teiler von $a$'' oder ``$g$ teilt $a$'',
gesucht ist also die grösste ganze Zahl $g$ derart, dass $g|a$ und $g|b$.

\begin{definition}
\label{buch:endliche-koerper:def:ggt}
Der grösste gemeinsame Teiler von $a$ und $b$ ist die grösste
ganze Zahl $g$, die sowohl $a$ als auch $b$ teilt: $g|a$ und
$g|b$.
\index{grösster gemeinsamer Teiler}%
\index{ggT}%
\end{definition}

Zusätzlich findet der euklidische Algorithmus  ganze Zahlen $s$
\index{euklidischer Algorithmus}%
und $t$ derart, dass
\[
sa + tb = g.
\]
In diesem Abschnitt soll der Algorithmus zunächst für ganze Zahlen
vorgestellt werden, bevor er auf Polynome verallgemeinert und dann
in Matrixform niedergeschrieben wird.
Die Matrixform ermöglicht, einfach zu implementierende iterative
Algorithmen für die Zahlen $s$ und $t$ und später auch für die
Berechnung des kleinsten gemeinsamen Vielfachen zu finden.

%
% Der euklidische Algorithmus für ganze Zahlen
%
\subsection{Grösster gemeinsamer Teiler ganzer Zahlen}
Gegeben sind zwei ganze Zahlen $a$ und $b$.
Wir dürfen annehmen, dass $a\ge b$.
Gesucht ist der grösste gemeinsame Teiler $g$ von $a$ und $b$.

Ist $b|a$, dann ist offenbar $b$ der grösste gemeinsame Teiler von $a$
und $b$.
Im Allgemeinen wird der grösste gemeinsame Teiler aber kleiner sein.
Wir teilen daher $a$ durch $b$, was nur mit Rest möglich ist.
Es gibt ganze Zahlen $q$, der Quotient, und $r$, der Rest, derart, dass
\begin{equation}
a = qb+ r
\qquad \Rightarrow \qquad
r = a - qb.
\label{lifting:euklid:raqb}
\end{equation}
Nach Definition des Restes ist $r < b$.
Da der grösste gemeinsame Teiler sowohl $a$ als auch $b$ teilt, muss er
wegen~\eqref{lifting:euklid:raqb} auch $r$ teilen.
Somit haben wir das Problem, den grössten gemeinsamen Teiler von $a$ und
$b$ zu finden, auf das ``kleinere'' Problem zurückgeführt, den grössten
gemeinsamen Teiler von $b$ und $r$ zu finden.

Um den eben beschriebenen Schritt zu wiederholen, wählen wir die folgende
Notation.
Wir schreiben $a_0=a$ und $b_0=b$.
Im ersten Schritt finden wird $q_0$ und $r_0$ derart,
dass $a_0-q_0b_0 = r_0$.
Dann setzen wir $a_1=b_0$ und $b_1=r_0$.
Mit $a_1$ und $b_1$ wiederholen wir den Divisionsschritt, der einen
neuen Quotienten $q_1$ und einen neuen Rest $r_1$ liefert mit $a_1-q_1b_1=r_1$.
So entstehen vier Folgen von Zahlen $a_k$, $b_k$, $q_k$ und $r_k$ derart,
dass in jedem Schritt gilt
\begin{align*}
a_k - q_kb_k &= r_k
&&\wedge &
g&|a_k
&&\wedge &
g&|b_k
&&\wedge &
a_k &= b_{k-1}
&&\wedge &
b_k = r_{k-1}.
\end{align*}
Der Algorithmus bricht im Schritt $n$ ab, wenn $r_{n+1}=0$.
Der letzte nicht verschwindende Rest $r_n$ muss daher der grösste gemeinsame
Teiler $g$ von $a$ und $b$ sein: $g=r_n$.

\begin{beispiel}
\label{buch:endlichekoerper:beispiel1}
Wir bestimmen den grössten gemeinsamen Teiler von $76415$ und $23205$
mit Hilfe des eben beschriebenen Algorithmus.
Wir schreiben die gefundenen Zahlen in eine Tabelle:
\begin{center}
\renewcommand{\arraystretch}{1.1}
\begin{tabular}{|>{$}r<{$}|>{$}r<{$}|>{$}r<{$}|>{$}r<{$}|>{$}r<{$}|}
\hline
k&  a_k&  b_k&   q_k&  r_k\\
\hline
0&76415&23205&     3&6800\\
1&23205& 6800&     3&2805\\
2& 6800& 2805&     2&1190\\
3& 2805& 1190&     2& 425\\
4& 1190&  425&     2& 340\\
5&  425&  340&     1&  85\\
6&  340&   85&     4&   0\\
\hline
\end{tabular}
\end{center}
Der Algorithmus bricht also mit dem letzten Rest $r_n=85$ ab, dies
ist der grösste gemeinsame Teiler.
\end{beispiel}

Die oben protokollierten Werte von $q_k$ werden für die Bestimmung
des grössten gemeinsamen Teilers nicht benötigt.
Wir können sie aber verwenden, um die Zahlen $s$ und $t$ zu bestimmen.

\begin{beispiel}
Wir drücken die Reste im obigen Beispiel durch die Zahlen $a_k$, $b_k$ und
$q_k$ aus und setzen sie in den Ausdruck $g=a_5-q_5b_5$ ein, bis wir
einen Ausdruck in $a_0$ und $b_0$ für $g$ finden:
\[
\begin{aligned}
r_5&=a_5-q_5 b_5=a_5-1\cdot b_5& g &= a_5 - 1 \cdot b_5 = b_4 - 1 \cdot r_4
\\
r_4&=a_4-q_4 b_4=a_4-2\cdot b_4&   &= b_4 - (a_4 -2b_4) 
                                    = -a_4 +3b_4 = -b_3 + 3r_3
\\
r_3&=a_3-q_3 b_3=a_3-2\cdot b_3&   &= -b_3 + 3(a_3-2b_3)
                                    = 3a_3 - 7b_3 = 3b_2 -7r_2
\\
r_2&=a_2-q_2 b_2=a_2-2\cdot b_2&   &= 3b_2 -7(a_2-2b_2)
                                    = -7a_2 + 17b_2 = -7b_1 + 17r_1
\\
r_1&=a_1-q_1 b_1=a_1-3\cdot b_1&   &= -7b_1 + 17(a_1-3b_1)
                                    = 17a_1 - 58b_1 = 17 b_0 - 58 r_0
\\
r_0&=a_0-q_0 b_0=a_0-3\cdot b_0&   &= 17b_0 - 58(a_0t-3b_0)
                                    = -58a_0+191b_0
\end{aligned}
\]
Tatsächlich gilt
\[
-58\cdot 76415 + 191 \cdot 23205 = 85,
\]
die Zahlen $t=-58$ und $s=191$ sind also genau die eingangs versprochenen
Faktoren.
\end{beispiel}

%
% Matrixschreibeweise für den euklidischen Algorithmus
%
\subsection{Matrixschreibweise
\label{buch:endlichekoerper:subsection:matrixschreibweise}}
Die Durchführung des euklidischen Algorithmus lässt sich besonders elegant
in Matrixschreibweise dokumentieren.
In jedem Schritt arbeitet man mit zwei ganzen Zahlen $a_k$ und $b_k$, die wir
als zweidimensionalen Spaltenvektor betrachten können.
Der Algorithmus macht aus $a_k$ und $b_k$ die neuen Zahlen
$a_{k+1} = b_k$ und $b_{k+1} = r_k = a_k - q_kb_k$, dies
kann man als die Matrixoperation
\[
\begin{pmatrix} a_{k+1} \\ b_{k+1} \end{pmatrix}
=
\begin{pmatrix} b_k \\ r_k \end{pmatrix}
=
\begin{pmatrix} 0 & 1 \\ 1 & -q_k \end{pmatrix}
\begin{pmatrix} a_{k} \\ b_{k} \end{pmatrix}
\]
schreiben.
Der Algorithmus bricht ab, wenn die zweite Komponente des Vektors $=0$ ist,
in der ersten steht dann der grösste gemeinsame Teiler.
Hier die Durchführung des Algorithmus in Matrixschreibweise:
\begin{align*}
\begin{pmatrix} 23205 \\ 6800 \end{pmatrix}
&=
\begin{pmatrix} 0&1\\1&-3 \end{pmatrix}
\begin{pmatrix} 76415 \\ 23205 \end{pmatrix}
\\
\begin{pmatrix} 6800 \\ 2805 \end{pmatrix}
&=
\begin{pmatrix} 0&1\\1&-3 \end{pmatrix}
\begin{pmatrix} 23205 \\ 6800 \end{pmatrix}
\\
\begin{pmatrix} 2805 \\ 1190 \end{pmatrix}
&=
\begin{pmatrix} 0&1\\1&-2 \end{pmatrix}
\begin{pmatrix} 6800 \\ 2805 \end{pmatrix}
\\
\begin{pmatrix} 1190 \\ 425 \end{pmatrix}
&=
\begin{pmatrix} 0&1\\1&-2 \end{pmatrix}
\begin{pmatrix} 2805 \\ 1190 \end{pmatrix}
\\
\begin{pmatrix} 425 \\ 340 \end{pmatrix}
&=
\begin{pmatrix} 0&1\\1&-2 \end{pmatrix}
\begin{pmatrix} 1190 \\ 425 \end{pmatrix}
\\
\begin{pmatrix} 340 \\ 85 \end{pmatrix}
&=
\begin{pmatrix} 0&1\\1&-1 \end{pmatrix}
\begin{pmatrix} 425 \\ 340 \end{pmatrix}
\\
\begin{pmatrix} 85 \\ 0 \end{pmatrix}
&=
\begin{pmatrix} 0&1\\1&-4 \end{pmatrix}
\begin{pmatrix} 340 \\ 85 \end{pmatrix}
=
\begin{pmatrix}g\\0\end{pmatrix}.
\end{align*}

\begin{definition}
Wir kürzen
\[
Q(q_k) = \begin{pmatrix} 0 & 1 \\ 1 & -q_k \end{pmatrix}
\]
ab.
\end{definition}

Mit dieser Definition lässt sich der euklidische Algorithmus wie folgt
beschreiben.

\begin{algorithmus}[Euklid]
\label{lifting:euklid}
Der Algorithmus operiert auf zweidimensionalen Vektoren
$x\in\mathbb Z^2$ 
wie folgt:
\begin{enumerate}
\item Initialisiere  den Vektor mit den ganzen Zahlen $a$ und $b$:
$\displaystyle x = \begin{pmatrix}x_1\\x_2\end{pmatrix}=\begin{pmatrix}a\\b\end{pmatrix}$
\item Bestimme den Quotienten $q$ als die grösste ganze Zahl,
für die $qx_2\le x_1$ gilt.
\item Berechne den neuen Vektor als $Q(q)x$.
\item Wiederhole Schritte 2 und 3 bis die zweite Komponente des Vektors
verschwindet.
Die erste Komponente ist dann der gesuchte grösste gemeinsame Teiler.
\end{enumerate}
\end{algorithmus}

Auch die Berechnung der Zahlen $s$ und $t$ lässt sich jetzt leichter verstehen.
Nach Algorithmus~\ref{lifting:euklid} ist
\[
\begin{pmatrix} g \\ 0 \end{pmatrix}
=
Q(q_n)Q(q_{n-1})\cdots Q(q_0)
\begin{pmatrix} a \\ b \end{pmatrix}.
\]
Schreiben wir $Q=Q(q_n)Q(q_{n-1})\cdots Q(q_0)$, dann enthält die Matrix
$Q$ in der erste Zeile die ganzen Zahlen $s$ und $t$, mit denen sich der
grösste gemeinsame Teiler aus $a$ und $b$ darstellen lässt:
\[
Q =
\begin{pmatrix}
s&t\\
q_{21}&q_{22}
\end{pmatrix}
\qquad\Rightarrow\qquad
\bigg\{
\quad
\begin{aligned}
g&=sa+tb\\
0&=q_{21}a+q_{22}b.
\end{aligned}
\]

\begin{beispiel}
Wir verifizieren die Behauptung durch Nachrechnen:
\begin{align*}
Q
&=
\begin{pmatrix} 0&1 \\ 1&-q_n\end{pmatrix}
\begin{pmatrix} 0&1 \\ 1&-q_{n-1}\end{pmatrix}
\cdots
\begin{pmatrix} 0&1 \\ 1&-q_{0}\end{pmatrix}
\\
&=
\underbrace{
\begin{pmatrix} 0&1 \\ 1& -4 \end{pmatrix}
\begin{pmatrix} 0&1 \\ 1& -1 \end{pmatrix}
}_{}
\underbrace{
\begin{pmatrix} 0&1 \\ 1& -2 \end{pmatrix}
\begin{pmatrix} 0&1 \\ 1& -2 \end{pmatrix}
}_{}
\underbrace{
\begin{pmatrix} 0&1 \\ 1& -2 \end{pmatrix}
\begin{pmatrix} 0&1 \\ 1& -3 \end{pmatrix}
}_{}
\begin{pmatrix} 0&1 \\ 1& -3 \end{pmatrix}
\\
&=
\underbrace{
\begin{pmatrix} 1 & -1 \\ -4 &  5 \end{pmatrix}
\begin{pmatrix} 1 & -2 \\ -2 &  5 \end{pmatrix}
}_{}
\underbrace{
\begin{pmatrix} 1 & -2 \\ -3 &  7 \end{pmatrix}
\begin{pmatrix} 0 &  1 \\  1 & -3 \end{pmatrix}
}_{}
\\ &=
\begin{pmatrix}  3 &  -7 \\ -14 &  33 \end{pmatrix}
\begin{pmatrix} -3 &  10 \\   7 & -23 \end{pmatrix}
=
\begin{pmatrix} -58 & 191 \\ 273 & -899 \end{pmatrix}.
%(%i9) Q6 . Q5
%                                 [  1   - 1 ]
%(%o9)                            [          ]
%                                 [ - 4   5  ]
%(%i10) Q4 . Q3
%                                 [  1   - 2 ]
%(%o10)                           [          ]
%                                 [ - 2   5  ]
%(%i11) Q2 . Q1
%                                 [  1   - 3 ]
%(%o11)                           [          ]
%                                 [ - 2   7  ]
%(%i12) Q6 . Q5 . Q4 . Q3
%                                 [  3    - 7 ]
%(%o12)                           [           ]
%                                 [ - 14  33  ]
%(%i13) Q2 . Q1 . Q0
%                                 [ - 3   10  ]
%(%o13)                           [           ]
%                                 [  7   - 23 ]
%(%i14) Q6 . Q5 . Q4 . Q3 . Q2 . Q1 . Q0
%                                [ - 58   191  ]
%(%o14)                          [             ]
%                                [ 273   - 899 ]
\end{align*}
In der zweiten Zeile findet man Zahlen, die $a$ und $b$ zu 0 kombinieren:
\[
273 \cdot 76415 - 899 \cdot 23205 = 0,
\]
in der ersten stehen die Zahlen $s=-58$ und $t=191$ und tatsächlich
ergibt
\[
ta+sb = -58\cdot 76415  + 191\cdot 23205 = 85 = g
\]
den grössten gemeinsamen Teiler von 76415 und 23205.
\end{beispiel}

Die Wirkung der Matrix
\[
Q(q) = \begin{pmatrix} 0 & 1 \\ 1 & -q \end{pmatrix}
\]
lässt sich mit genau einer Multiplikation und einer Addition
berechnen.

%
% Vereinfachte Durchführung des euklidischen Algorithmus
%
\subsection{Iterative Durchführung des erweiterten euklidischen Algorithmus
\label{buch:endlichekoerper:subsection:erweitertereuklidischeralgo}}
Die Durchführung des euklidischen Algorithmus mit Hilfe der Matrizen
$Q(q_k)$ ist etwas unhandlich.
In diesem Abschnitt sollen die Matrizenprodukte daher in einer Form
dargestellt werden, die leichter als Programm zu implementieren ist.

In Abschnitt~\ref{buch:endlichekoerper:subsection:matrixschreibweise}
wurde gezeigt, dass das Produkt der aus den Quotienten $q_k$ gebildeten
Matrizen $Q(q_k)$ berechnet werden muss.
Im Folgenden soll ein iterativer Algorithmus zu seiner Berechnung
hergeleitet werden.
Dazu beachten wir zunächst, dass die Multiplikation mit der Matrix
$Q(q_k)$ die zweite Zeile in die erste Zeile verschiebt:
\[
Q(q_k)
\begin{pmatrix}
u&v\\
c&d\\
\end{pmatrix}
=
\begin{pmatrix}0&1\\1&-q_k\end{pmatrix}
\begin{pmatrix}
u&v\\
c&d
\end{pmatrix}
=
\begin{pmatrix}
c&d\\
u-q_kc&v-q_kd
\end{pmatrix}.
\]
Die Matrizen
\[
Q_k = Q(q_k)Q(q_{k-1})\cdots Q(q_0)
\]
haben daher jeweils für aufeinanderfolgende Werte von $k$ eine Zeile
gemeinsam.
Wir bezeichnen die Einträge der ersten Zeile der Matrix $Q_k$ mit
$c_k$ und $d_k$.
Es gilt dann
\[
Q_k
=
\begin{pmatrix}
c_{k}  &d_{k}  \\
c_{k+1}&d_{k+1}
\end{pmatrix}
=
Q(q_k)
\begin{pmatrix}
c_{k-1}&d_{k-1}\\
c_{k}  &d_{k}
\end{pmatrix}.
\]
Daraus ergeben sich die Rekursionsformeln
\begin{equation}
\begin{aligned}
c_{k+1}&=c_{k-1}-q_kc_k\\
d_{k+1}&=d_{k-1}-q_kd_k.
\end{aligned}
\label{buch:endlichekoerper:eqn:cdrekursion}
\end{equation}
Die Auswertung des Matrizenproduktes von links nach rechts beginnt mit
der Einheitsmatrix, es ist
\[
Q_0
=
Q(q_0) I
=
\begin{pmatrix}
0&1\\
1&-q_0
\end{pmatrix}
\begin{pmatrix}
1&0\\0&1\end{pmatrix},
\]
woraus man ablesen kann, dass
\begin{equation}
Q_{-1}
=
\begin{pmatrix}
c_{-1}&d_{-1}\\
c_0&d_0
\end{pmatrix}
=
\begin{pmatrix}
1&0\\
0&1
\end{pmatrix}
\label{buch:endlichekoerper:eqn:cdinitial}
\end{equation}
gesetzt werden muss.

Mit diesen Notationen kann man den Algorithmus jetzt in der früher
verwendeten Tabelle durchführen, die man um die zwei
Spalten $c_k$ und $d_k$ erweitert und die Werte in dieser
Spalte mit Hilfe der
Rekursionsformeln~\eqref{buch:endlichekoerper:eqn:cdrekursion}
aus den initialen Werten~\eqref{buch:endlichekoerper:eqn:cdinitial}
berechnet.

\begin{beispiel}
Wir erweitern das Beispiel von Seite~\pageref{buch:endlichekoerper:beispiel1}
zur Bestimmung des grössten gemeinsamen Teilers von $76415$ und $23205$
um die Spalten zur Berechnung der Koeffizienten $c_k$ und $d_k$.
Wir schreiben die gefundenen Zahlen in eine Tabelle:
\begin{center}
\label{buch:endlichekoerper:beispiel1erweitert}
\begin{tikzpicture}[>=latex,thick]
\fill[color=darkgreen!30] (1.8,-0.4) rectangle (3.5,0.4);
\node at (3.10,0) {$\displaystyle
\color{darkgreen}
\begin{pmatrix}
\phantom{-33}\,\,&\phantom{-233}\\
\phantom{-33}\,\,&\phantom{-233}
\end{pmatrix}\;=Q_2$};
\node at (0,0) {$\displaystyle
\renewcommand{\arraystretch}{1.1}
\begin{tabular}{|>{$}r<{$}|>{$}r<{$}|>{$}r<{$}|>{$}r<{$}|>{$}r<{$}|>{$}r<{$}>{$}r<{$}|}
\hline
k&   a_k&   b_k&    q_k&  r_k&     c_k&     d_k\\
\hline
 &      &      &       &     &       1&       0\\
0& 76415& 23205&      3& 6800&       0&       1\\
1& 23205&  6800&      3& 2805&       1&      -3\\
2&  6800&  2805&      2& 1190&      -3&      10\\
3&  2805&  1190&      2&  425&       7&     -23\\
4&  1190&   425&      2&  340&     -17&      56\\
5&   425&   340&      1&   85&      41&    -135\\
6&   340&    85&      4&    0&     -58&     191\\
7&    85&     0&       &     &     273&    -899\\
\hline
\end{tabular}$};
\end{tikzpicture}
\end{center}
Aus den letzten zwei Spalten der Tabelle kann man ablesen, dass
\begin{align*}
-58\cdot 76415 + 191\cdot 23205 &= 85\\
273\cdot 76415 - 899\cdot 23205 &= 0,
\end{align*}
wie erwartet.
Die gesuchten Zahlen $s$ und $t$ sind also $s=-58$ und $t=191$.
\end{beispiel}

Die Matrizen $Q_k$ kann man auch aus der Tabelle ablesen, sie bestehen
aus den vier Elementen in den Zeilen $k$ und $k+1$ in den
Spalten $c_k$ und $d_k$.
Die Matrix $Q_2$ ist beispielhaft grün hervorgehoben.
Auf jeder Zeile gilt $b_k = c_ka_0 + d_kb_0$, für $k>0$ ist dies
$c_ka_0+d_kb_0=r_{k-1}$.

Bis jetzt gingen wir immer davon aus, dass $a>b$ ist.
Dies ist jedoch nicht nötig, wie die Durchführung des Algorithmus
für das obige Beispiel mit vertauschten Werten von $a$ und $b$ zeigt.
Wir bezeichnen die Elemente zur Unterscheidung von der ursprünglichen
Durchführung mit einem Strich:
\begin{center}
\renewcommand{\arraystretch}{1.1}
\begin{tabular}{|>{$}r<{$}|>{$}r<{$}|>{$}r<{$}|>{$}r<{$}|>{$}r<{$}|>{$}r<{$}>{$}r<{$}|}
\hline
k&  a_k'&  b_k'&   q_k'&  r_k'&    c_k'&    d_k'\\
\hline
 &      &      &       &      &       1&       0\\
0& 23205& 76415&      0& 23205&       0&       1\\
1& 76415& 23205&      3&  6800&       1&       0\\
2& 23205&  6800&      3&  2805&      -3&       1\\
3&  6800&  2805&      2&  1190&      10&      -3\\
4&  2805&  1190&      2&   425&     -23&       7\\
5&  1190&   425&      2&   340&      56&     -17\\
6&   425&   340&      1&    85&    -135&      41\\
7&   340&    85&      4&     0&     191&     -58\\
8&    85&     0&       &      &    -899&     273\\
\hline
\end{tabular}
\end{center}
Da für $a<b$ der erste Quotient $q_0'=0$ ist, werden die ersten neuen
Elemente $c_1'=1=d_0$ und $d_1'=0=c_0$ sein.
Die nachfolgenden Quotienten sind genau die gleichen, also $q_k = q_{k+1}'$
und damit werden auch
\[
c_{k}=d_{k+1}' \qquad\text{und}\qquad d_{k} = c_{k+1}'
\]
sein.
Man findet also die gleichen Einträge in einer Tabelle, die eine Zeile
mehr hat und in der die letzten zwei Spalten gegenüber der ursprünglichen
Tabelle vertauscht wurden.

%
% Der euklidische Algorithmus für Polynome
%
\subsection{Grösster gemeinsare Teiler von Polynomen}
Der Ring $\mathbb{Q}[X]$ der Polynome in der Variablen $X$ mit rationalen
Koeffizienten\footnote{Es kann auch ein beliebiger anderer Körper für
die Koeffizienten verwendet werden.
Es gelten sogar ähnlich interessante Gesetzmässigkeiten, wenn man für
die Koeffizienten ganze Zahlen zulässt.
Dann wird das Problem der Faktorisierung allerdings verkompliziert 
durch das Problem der Teilbarkeit der Koeffizienten.
Dieses Problem entfällt, wenn man die Koeffizienten aus einem
Bereich wählt, in dem Teilbarkeit kein Problem ist, also in einem Körper.}
verhält
sich bezüglich Teilbarkeit ganz ähnlich wie die ganzen Zahlen.
Insbesondere ist der euklidische Algorithmus genauso wie die
Matrixschreibweise auch für Polynome durchführbar.

\begin{beispiel}
\label{buch:endlichekoerper:eqn:polynomggt}
Wir berechnen als Beispiel den grössten gemeinsamen Teiler 
der Polynome
\[
a = X^4 - 2X^3 -7 X^2 + 8X + 12,
\qquad
b = X^4 + X^3 -7X^2 -X + 6.
\]
Wir erstellen wieder die Tabelle der Reste
\begin{center}
\renewcommand{\arraystretch}{1.4}
\begin{tabular}{|>{$}r<{$}|>{$}r<{$}|>{$}r<{$}|>{$}r<{$}|>{$}r<{$}|}
\hline
k&  a_k&  b_k&   q_k&  r_k\\
\hline
0& X^4 - 2X^3 -7 X^2 + 8X + 12& X^4 + X^3 -7X^2 -X + 6&       1&-3X^3+9X+6\\
1&X^4+X^3-7X^2-X+6            &-3X^3+9X+6             &-\frac13X-\frac13&-4X^2+4X+8\\
2&-3X^3+9X+6      &-4X^2+4X+8& \frac34 X + \frac34& 0\\
\hline
\end{tabular}
\end{center}
Daraus kann man ablesen, dass $-4x^2+4x+8$ grösster gemeinsamer Teiler ist.
Normiert auf einen führenden Koeffizienten $1$ ist dies das Polynom
$x^2-x+2=(x+2)(x-1)$.

Wir berechnen auch noch die Polynome $s$ und $t$.
Dazu müssen wir die Matrizen $Q(q_k)$ miteinander multiplizieren:
\begin{align*}
Q
&=Q(q_2) Q(q_1) Q(q_0)
\\
&=
\begin{pmatrix} 0 & 1 \\ 1 & -\frac34(X+1) \end{pmatrix}
\begin{pmatrix} 0 & 1 \\ 1 & \frac13(X+1) \end{pmatrix}
\begin{pmatrix} 0 & 1 \\ 1 & -1 \end{pmatrix}
\\
&=
%                        [     x   1         2   x    ]
%                        [     - + -         - - -    ]
%                        [     3   3         3   3    ]
%(%o22)                  [                            ]
%                        [     2            2         ]
%                        [    x     x   3  x    x   3 ]
%                        [ (- --) - - + -  -- - - - - ]
%                        [    4     2   4  4    4   2 ]
\begin{pmatrix}
\frac13(X+1)&-\frac13(X-2)\\
-\frac14(X^2+2X-3)&\frac14(X^2-X-6)
\end{pmatrix}.
\end{align*}
In der ersten Zeile finden wir die Polynome $t(X)$ und $s(X)$, mit denen
\begin{align*}
ta+sb
&=
\frac13(X+1)
(X^4-2X^3-7X^2+8X+12)
-\frac13(X-2)
(X^4+X^3-7X^2-X+6)
\\
&=
-4X^2+4X+8,
\end{align*}
und dies ist tatsächlich der gefundene grösste gemeinsame Teiler.
Die zweite Zeile von $Q$ gibt uns die Faktoren, mit denen
$a$ und $b$ gleich werden:
\begin{align*}
q_{21}a+q_{22}b
&=
-\frac14(X^2+2X-3)
(X^4-2X^3-7X^2+8X+12)
+\frac14(X^2-X-6)
(X^4+X^3-7X^2-X+6)
\\
&=0.
\qedhere
\end{align*}
Man kann natürlich den grössten gemeinsamen Teiler auch mit Hilfe einer
Faktorisierung der Polynome $a$ und $b$ finden.
Um die Übersicht zu erleichtern, werden gleiche Faktoren jeweils untereinander
geschrieben und fehlende Faktoren als Platzhalter in grau dargestellt.
\def\grau#1{\bgroup\color{gray!50}#1\egroup}
\begin{align*}
&\text{Faktorisierung von $a$:}&
a    &=      (X-3)      (X-2)\grau{(X-1)}     (X+1)       (X+2) \grau{(X+3)}\\
&\text{Faktorisierung von $b$:}&
b    &=\grau{(X-3)}     (X-2)      (X-1)      (X+1) \grau{(X+2)}      (X+3) \\
&\text{gemeinsame Faktoren:}&
g    &=\grau{(X-3)}     (X-2)\grau{(X-1)}     (X+1) \grau{(X+2)}\grau{(X+3)}
    = X^2 -X + 2\\
&&
v=a/g  &=    (X-3)\grau{(X-2)      (X-1)      (X+1)}      (X+2) \grau{(X+3)}
    = X^2-X-6 \\
&&
u=b/g&=\grau{(X-3)      (X-2)}     (X-1)\grau{(X+1)       (X+2)}      (X+3)
    = X^2+2X-3
\end{align*}
Aus den letzten zwei Zeilen folgt
$ua-vb = ab/g - ab/g = 0$, wie erwartet.
\end{beispiel}

%
% Das kleinste gemeinsame Vielfache
%
\subsection{Das kleinste gemeinsame Vielfache
\label{buch:subsection:daskgv}}
\index{kleinstes gemeinsames Vielfaches}%
\index{kgV}%
Das kleinste gemeinsame Vielfache zweier Zahlen $a$ und $b$ ist
\[
\operatorname{kgV}(a,b)
=
\frac{ab}{\operatorname{ggT}(a,b)}.
\]
Wir suchen nach einen Algorithmus, mit dem man das kleinste gemeinsame
Vielfache effizient berechnen kann.

Die Zahlen $a$ und $b$ sind beide Vielfache des grössten gemeinsamen
Teilers $g=\operatorname{ggT}(a,b)$.
Es gibt daher Zahlen $u$ und $v$ derart, dass $a=ug$ und $b=vg$.
Wenn $t$ ein gemeinsamer Teiler von $u$ und $v$ ist, dann ist $tg$ ein
grösserer gemeinsamer Teiler von $a$ und $b$.
Dies kann nicht sein, also müssen $u$ und $v$ teilerfremd sein.
Das kleinste gemeinsame Vielfache von $a$ und $b$ ist dann $ugv=av=ub$.
Die Bestimmung des kleinsten gemeinsamen Vielfachen ist also gleichbedeutend
mit der Bestimmung der Zahlen $u$ und $v$.

Die definierende Eigenschaften von $u$ und $v$ kann man in Matrixform als
\index{Matrixform des kgV-Algorithmus}%
\begin{equation}
\begin{pmatrix}
a\\b
\end{pmatrix}
=
\underbrace{
\begin{pmatrix}
u&?\\
v&?
\end{pmatrix}}_{\displaystyle =K}
\begin{pmatrix}
\operatorname{ggT}(a,b)\\ 0
\end{pmatrix}
\label{buch:eindlichekoerper:eqn:uvmatrix}
\end{equation}
geschrieben werden, wobei wir die Matrixelemente $?$ nicht kennen.
Diese Elemente müssen wir auch nicht kennen, um $u$ und $v$ zu bestimmen.

Bei der Bestimmung des grössten gemeinsamen Teilers wurde der Vektor auf
der rechten Seite von~\eqref{buch:eindlichekoerper:eqn:uvmatrix} bereits
gefunden.
Die Matrizen $Q(q_i)$, die die einzelne Schritte des euklidischen
Algorithmus beschreiben, ergeben ihn als
\[
\begin{pmatrix}
\operatorname{ggT}(a,b)\\0
\end{pmatrix}
=
Q(q_n)Q(q_{n-1}) \cdots Q(q_1)Q(q_0)
\begin{pmatrix}a\\b\end{pmatrix}.
\]
Indem wir die Matrizen $Q(q_n)$ bis $Q(q_0)$ auf die linke Seite der
Gleichung schaffen, erhalten wir
\[
\begin{pmatrix}a\\b\end{pmatrix}
=
Q(q_0)^{-1}
Q(q_1)^{-1}
\cdots
Q(q_{n-1})^{-1}
Q(q_n)^{-1}
\begin{pmatrix}\operatorname{ggT}(a,b)\\0\end{pmatrix}.
\]
Eine mögliche Lösung für die Matrix $K$ in
\eqref{buch:eindlichekoerper:eqn:uvmatrix}
ist daher die Matrix
\[
K
=
Q(q_0)^{-1}
Q(q_1)^{-1}
\cdots
Q(q_{n-1})^{-1}
Q(q_n).
\]
Insbesondere ist die Matrix $K$ die Inverse der früher gefundenen
Matrix $Q$.

Die Berechnung der Matrix $K$ als Inverse von $Q$ ist nicht schwierig,
aber eingebaut in den bereits etablierten iterativen Prozess fällt
sie noch leichter.
Genauso wie es möglich war, das Produkt $Q$ der Matrizen
$Q(q_k)$ iterativ zu bestimmen, muss es auch eine Rekursionsformel
für das Produkt der inversen Matrizen $Q(q_k)^{-1}$ geben.

Schreiben wir die gesuchte Matrix 
\[
K_k
=
Q(q_0)^{-1}\cdots Q(q_{k-1})^{-1}
=
\begin{pmatrix}
e_k & e_{k-1}\\
f_k & f_{k-1}
\end{pmatrix},
\]
dann kann man $K_k$ durch die Rekursion
\begin{equation}
K_{k+1}
=
K_{k} Q(q_k)^{-1} 
=
K_k K(q_k)
\qquad\text{mit}\qquad
K_0 = \begin{pmatrix}1&0\\0&1\end{pmatrix} = I
\label{buch:endlichekoerper:eqn:kgvrekursion}
\end{equation}
berechnen.
Die Inverse von $Q(q)$ ist
\[
K(q)
=
Q(q)^{-1}
=
\frac{1}{\det Q(q)}
\begin{pmatrix}
q&1\\
1&0
\end{pmatrix}
\quad\text{denn}\quad
K(q)Q(q)
=
\begin{pmatrix}
q&1\\
1&0
\end{pmatrix}
\begin{pmatrix}
0&1\\
1&-q
\end{pmatrix}
=
\begin{pmatrix}
1&0\\
0&1
\end{pmatrix}.
\]
Da die zweite Spalte von $K(q)$ die erste Spalte einer Einheitsmatrix
ist, wird die zweite Spalte des Produktes $AK(q)$ immer die erste Spalte
von $A$ sein.
In $K_{k+1}$ ist daher nur die erste Spalte neu, die zweite Spalte ist
die erste Spalte von $K_k$.

Aus der Rekursionsformel \eqref{buch:endlichekoerper:eqn:kgvrekursion}
für die Matrizen $K_k$ kann man jetzt eine Rekursionsbeziehung
für die Folgen $e_k$ und $f_k$ ablesen, es gilt
\begin{align*}
e_{k+1} &= q_ke_k + e_{k-1} \\
f_{k+1} &= q_kf_k + f_{k-1}
\end{align*}
für $k=0,1,\dots ,n$.
Damit können $e_k$ und $f_k$ gleichzeitig mit den Zahlen $c_k$ und $d_k$
in einer Tabelle berechnet werden.

\begin{beispiel}
Wir erweitern das Beispiel von
Seite~\pageref{buch:endlichekoerper:beispiel1erweitert}
um die beiden Spalten zur Berechnung von $e_k$ und $f_k$:
\begin{center}
\renewcommand{\arraystretch}{1.1}
\begin{tabular}{|>{$}r<{$}|>{$}r<{$}|>{$}r<{$}|>{$}r<{$}|>{$}r<{$}|>{$}r<{$}>{$}r<{$}|>{$}r<{$}>{$}r<{$}|}
\hline
k&   a_k&   b_k&    q_k&  r_k&     c_k&     d_k&   e_k&   f_k\\
\hline
 &      &      &       &     &       1&       0&     0&     1\\
0& 76415& 23205&      3& 6800&       0&       1&     1&     0\\
1& 23205&  6800&      3& 2805&       1&      -3&     3&     1\\
2&  6800&  2805&      2& 1190&      -3&      10&    10&     3\\
3&  2805&  1190&      2&  425&       7&     -23&    23&     7\\
4&  1190&   425&      2&  340&     -17&      56&    56&    17\\
5&   425&   340&      1&   85&      41&    -135&   135&    41\\
6&   340&    85&      4&    0&     -58&     191&   191&    58\\
7&    85&     0&       &     &     273&    -899&   899&   273\\
\hline
\end{tabular}
\end{center}
Der grösste gemeinsame Teiler ist $\operatorname{ggT}(a,b)=85$.
Aus der letzten Zeile der Tabelle kann man jetzt die Zahlen $u=e_7=899$
und $v=f_7=273$ ablesen, und tatsächlich ist
\[
a=76415 = 899\cdot 85
\qquad\text{und}\qquad
b=23205 = 273 \cdot 85.
\]
Daraus kann man dann auch das kleinste gemeinsame Vielfache ablesen, es ist
\[
\operatorname{kgV}(a,b)
=
\operatorname{kgV}(76415,23205)
=
\left\{
\begin{aligned}
ub
&=
899\cdot 23205\\
va
&=
273\cdot 76415
\end{aligned}
\right\}
=
20861295.
\qedhere
\]
\end{beispiel}

\subsection{Kleinstes gemeinsames Vielfaches von Polynomen}
Der erweiterte Algorithmus kann auch dazu verwendet werden,
das kleinste gemeinsame Vielfache zweier Polynome zu berechnen.
Dies wird zum Beispiel bei der Decodierung des Reed-Solomon-Codes in
Kapitel~\ref{chapter:reedsolomon} verwendet.

Im Beispiel auf Seite~\pageref{buch:endlichekoerper:eqn:polynomggt}
wird der grösste gemeinsame Teiler der Polynome
\[
a
=
X^4 - 2X^3 -7 X^2 + 8X + 12,
\qquad
b = X^4 + X^3 -7X^2 -X + 6
\]
berechnet.
Dies kann jetzt erweitert werden für die Berechnung des kleinsten
gemeinsamen Vielfachen.
\index{kleinstes gemeinsames Vielfaches von Polynomen}%
\index{kgV von Polynomen}%

\begin{beispiel}
Die Berechnungstabelle nur für die Spalten $e_k$ und $f_k$ ergibt
\begin{center}
\renewcommand{\arraystretch}{1.4}
\begin{tabular}{|>{$}r<{$}|>{$}r<{$}|>{$}r<{$}>{$}r<{$}|}
\hline
k&   q_k&  e_k&  f_k\\
\hline
 &                 &                           0&                           1\\
0&                1&                           1&                           0\\
1&-\frac13X-\frac13&                           1&                           1\\
2& \frac34X+\frac34&           -\frac13X+\frac23&           -\frac13X-\frac13\\
 &                 &-\frac14X^2+\frac14X+\frac32&-\frac14X^2-\frac12X+\frac34\\
\hline
\end{tabular}
\end{center}
Daraus kann man ablesen, dass
\[
u
=
-\frac14X^2+\frac14X+\frac32
\qquad\text{und}\qquad
v
=
-\frac14X^2-\frac12X+\frac34.
\]
Daraus ergibt sich das kleinste gemeinsame Vielfache auf zwei verschiedene Weisen:
\[
\operatorname{ggT}(a,b)
=
\left\{
\begin{aligned}
\textstyle
(-\frac14X^2+\frac14X+\frac32)&\cdot(X^4 - 2X^3 -7 X^2 + 8X + 12)
\\
\textstyle
(-\frac14X^2-\frac12X+\frac34)&\cdot(X^4 + X^3 -7X^2 -X + 6)
\end{aligned}
\right\}
=
-\frac14X^6+\frac72X^4-\frac{49}4X^2+9.
\]
Die beiden Berechnungsmöglichkeiten stimmen wie erwartet überein.
\end{beispiel}

\subsection{Anwendung: Decodierung des Reed-Solomon-Codes}
Der Reed-Solomon-Code verwendet Polynome zur Codierung der Daten,
\index{Reed-Solomon-Code}%
dies wird in Kapitel~\ref{chapter:reedsolomon} im Detail beschrieben.
Bei der Decodierung muss der Faktor $u$ für zwei gegebene Polynome
$n(X)$ und $r(X)$ bestimmt werden.
Allerdings ist das Polynom $r(X)$ nicht vollständig bekannt, nur die 
ersten paar Koeffizienten sind gegeben.
Dafür weiss man zusätzlich, wieviele Schritte genau der Euklidische
Algorithmus braucht.
Daraus lässt sich genügend Information gewinnen, um die Faktoren $u$
und $v$ zu bestimmen.
Das Video \url{https://youtu.be/uOLW43OIZJ0} von Edmund Weitz
\index{Weitz, Edmund}
erklärt die Theorie hinter dieser Teilaufgabe anhand von Beispielen.

\begin{beispiel}
Wir berechnen also die Faktoren $u$ und $v$ für die beiden Polynome
\begin{align*}
n(X)
&=
X^{12}+12
\\
r(X)
&=
7 X^{11} + 4 X^{10} + X^9 + 12 X^8 + 2 X^7 + 12 X^6 + w(X)
\end{align*}
in $\mathbb{F}_{13}[X]$, wobei $w(X)$ ein unbekanntes Polynom vom Grad $5$ ist. 
Man weiss zusätzlich noch, dass der euklidische Algorithmus genau drei
Schritte braucht, es gibt also genau drei Quotienten, die in die
Berechnung der Zahlen $e_k$ und $f_k$ einfliessen.

In den folgenden Zeilen wird der euklische Algorithmus für die Polynome
$n(X)$ und $r(X)$ durchgeführt.
Im ersten Schritt ist der Quotient
$n(X) / r(X)$ zu bestimmen, der Grad $1$ haben muss.
\begin{align*}
a_0=n(X)           &= X^{12}+12
\\
b_0=r(X)           &= 7 X^{11} + 4 X^{10} + X^9 + 12 X^8 + 2 X^7 + 12 X^6 + \dots
\\
q_0                &= 2X+10
\\
r_0 = a_0-b_0\cdot q_0 &= 10X^{10} + 5X^9 + 6X^8 + 8X^7 + \dots
\\
a_1 &= 7 X^{11} + 4 X^{10} + X^9 + 12 X^8 + 2 X^7 + 12 X^6 + \dots
\\
b_1 &= 10X^{10} + 5X^9 + 6X^8 + 8X^7 + \dots
\\
q_1 &= 2X+2
\\
r_1 = a_1 - b_1q_1 &= 5X^9 + 10 X^8 + \dots
\\
a_2 &= 10X^{10} + 5X^9 + 6X^8 + 8X^7 + \dots
\\
b_2 &= 5X^9 + 10 X^8 + \dots
\\
q_2 &= 2X+10.
\end{align*}
Aus den Polynomen $q_k$ können jetzt die Faktoren $u$ und $v$
bestimmt werden:
\begin{center}
\begin{tabular}{|>{$}c<{$}|>{$}r<{$}|>{$}r<{$}|>{$}r<{$}|}
\hline
k&   q_k&               e_k&        f_k\\
\hline
 &      &                 0&          1\\
0& 2X+10&                 1&          0\\
1& 2X+2 &             2X+10&          1\\
2& 2X+10&        4X^2+11X+8&       2X+2\\
 &      & 8X^3+10X^2+11X+12& 4X^2+11X+8\\
\hline
\end{tabular}
\end{center}
Die Faktorisierung des Polynoms
\[
u
=
8X^3+10X^2+11X+12
\]
kann bestimmt werden, indem man alle Zahlen $1,2,\dots,12\in\mathbb{F}_{13}$
einsetzt.
Man findet so die Nullstellen $3$, $4$ und $8$, also muss das Polynom
$u$ faktorisiert werden können als
\[
u=
8(X-3)(X-4)(X-8)
=
8X^3 - 120X^2+544X-768
=
8X^3 +10X^2+11X+12.
\qedhere
\]
\end{beispiel}

