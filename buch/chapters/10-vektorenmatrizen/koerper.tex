%
% koerper.tex -- Definition eines Körpers
%
% (c) 2021 Prof Dr Andreas Müller, OST Ostschwêizer Fachhochschule
%
\subsection{Körper
\label{buch:subsection:koerper}}
Die Multiplikation ist in einer Algebra nicht immer umkehrbar.
Die Zahlenkörper von Kapitel~\ref{buch:chapter:zahlen} sind also
sehr spezielle Algebren, man nennt sie Körper.
In diesem Abschnitt sollen die wichtigsten Eigenschaften von Körpern
zusammengetragen werden.

\begin{definition}
Ein Körper $K$ ist ein additive Gruppe mit einer multiplikativen
Verknüpfung derart, dass $K^* = K \setminus \{0\}$ eine Gruppe bezüglich
der Multiplikation ist.
Ausserdem gelten die Distributivgesetze 
\[
(a+b)c = ac+bc
\qquad a,b,c\in K.
\]
\end{definition}

Ein Körper ist also ein Ring derart, dass die Einheitengruppe $K^*$ ist.

\begin{beispiel}
Die Menge $\mathbb{F}_2=\{0,1\}$ mit der Additions- und
Mutliplikationstabelle
\begin{center}
\begin{tabular}{|>{$}c<{$}|>{$}c<{$}>{$}c<{$}|}
\hline
+&0&1\\
\hline
0&0&1\\
1&1&0\\
\hline
\end{tabular}
\qquad
\qquad
\qquad
\begin{tabular}{|>{$}c<{$}|>{$}c<{$}>{$}c<{$}|}
\hline
\cdot&0&1\\
\hline
0&0&0\\
1&0&1\\
\hline
\end{tabular}
\end{center}
ist der kleinste mögliche Körper.
\end{beispiel}

\begin{beispiel}
Die Menge der rationalen Funktionen
\[
\mathbb{Q}(z)
=
\biggl\{
f(z)
=
\frac{p(z)}{q(z)}
\,
\bigg|
\,
\begin{minipage}{5.5cm}
\raggedright
$p(z), q(z)$ sind Polynome mit rationalen Koeffizienten, $q(z)\ne 0$
\end{minipage}
\,
\biggr\}
\]
ist ein Körper.
\end{beispiel}

Kapitel~\ref{buch:chapter:endliche-koerper} wird eingehender weitere
Körper studieren.
Diese endlichen Körper sind vor allem in Kryptographie und Codierungstheorie
sehr nützlich.

