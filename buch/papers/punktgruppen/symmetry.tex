\section{Symmetrie}
\rhead{Symmetrie}
Das Wort Symmetrie ist sehr alt und hat sich seltsamerweise von seinem
ursprünglichen griechischen Wort \(\mathrm{\Sigma\upsilon\mu\mu\varepsilon\tau\rho\iota\alpha}\)\footnote{\emph{Symmetr\'ia}: ein gemeinsames Mass habend, gleichmässig,verhältnismässig} fast nicht verändert.
In der Alltagssprache mag es ein locker definierter Begriff sein, in der Mathematik hat Symmetrie jedoch eine sehr präzise Bedeutung.
\begin{definition}[Symmetrie]
  Ein mathematisches Objekt wird als symmetrisch bezeichnet, wenn es unter einer bestimmten Operation invariant ist.
\end{definition}
Die intuitivsten Beispiele kommen aus der Geometrie, daher werden wir mit einigen geometrischen Beispielen beginnen.
Wie wir jedoch später sehen werden, ist das Konzept der Symmetrie eigentlich viel allgemeiner.

\begin{figure}
  \centering
  \includegraphics{papers/punktgruppen/figures/symmetric-shapes}
  \caption{
    Beispiele für geometrisch symmetrische Formen.
    \label{fig:punktgruppen:geometry-example}
  }
\end{figure}

\subsection{Geometrische Symmetrien}

In Abbildung \ref{fig:punktgruppen:geometry-example} haben wir einige Formen, die offensichtlich symmetrisch sind.
Zum Beispiel hat das Quadrat eine Gerade, an der es gespiegelt werden kann, ohne sein Aussehen zu verändern.
Regelmässige Polygone mit \(n\) Seiten sind auch gute Beispiele, um eine diskrete Rotationssymmetrie zu veranschaulichen, was bedeutet, dass eine Drehung um einen Punkt um einen bestimmten Winkel \(360^\circ/n\) die Figur unverändert lässt.
Das letzte Beispiel in Abbildung~\ref{fig:punktgruppen:geometry-example} rechts ist eine unendliche Rotationssymmetrie. Sie wird so genannt, weil es unendlich viele Werte für den Drehwinkel \(\alpha \in \mathbb{R}\) gibt, die die Form unverändert lassen.

Ein Objekt kann mehr als nur eine Symmetrie aufweisen.
Zum Beispiel kann das Quadrat in Abbildung \ref{fig:punktgruppen:geometry-example} nicht nur um \(\sigma\) sondern auch diagonal gespiegelt werden oder um \(90^\circ\) gedreht werden.
Fasst man die möglichen Symmetrien zusammen, entsteht eine Symmetriegruppe.

\begin{definition}[Symmetriegruppe]
\index{Symmetriegruppe}%
  Seien \(g\) und \(h\) umkehrbare Operationen, sogenannte Symmetrieoperationen, die ein mathematisches Objekt unverändert lassen.
  Die Komposition \(h\circ g\) definieren wir als die Anwendung der Operationen nacheinander.
  Alle möglichen Symmetrieoperationen bilden unter Komposition eine Gruppe, die Symmetriegruppe genannt wird.
\index{Komposition}%
\end{definition}

Eine Gruppe benötigt ausserdem auch zwingend ein neutrales Element, welches wir mit \(\mathds{1}\) bezeichnen.
\index{neutrales Element}%
\index{1@$\mathds{1}$}%
Die Anwendung der neutralen Operation ist gleichbedeutend damit, alles unverändert zu lassen.
Weiterhin muss in einer Gruppe für jede Operation \(g\) auch eine inverse Operation \(g^{-1}\) vorkommen, die rückgängig macht, was \(g\) getan hat.
Somit ist \(\mathds{1}\) auch äquivalent dazu, eine Operation und dann ihre Inverse anzuwenden.
 Die Definition der Symmetriegruppe ist mit der Kompositionsoperation gegeben, sie wird aber auch oft als Multiplikation geschrieben.
Das liegt daran, dass in manchen Fällen die Zusammensetzung algebraisch durch eine Multiplikation berechnet wird.
Die Verwendung einer multiplikativen Schreibweise ermöglicht es, einige Ausdrücke kompakter zu schreiben, z.B.
durch Verwendung von Potenzen \(r^n = r\circ r \circ \cdots r\circ r\) für eine wiederholte Komposition.
\index{Potenzen von Symmetrieoperationen}%
\index{wiederholte Komposition}%

\begin{definition}[Zyklische Untergruppe, Erzeuger]
\index{zyklische Gruppe}%
\index{Erzeuger}%
  Sei \(g\) ein Element einer Symmetriegruppe \(G\).
  Alle möglichen Kompositionen von \(g\) und \(g^{-1}\) bilden eine sogenannte zyklische Untergruppe von \(G\), wobei \(g\) Erzeuger der Untergruppe genannt wird.
  Die von \(g\) erzeugte Untergruppe \(\langle g \rangle = \{ g^k : k \in \mathbb{Z} \}\) wird mit spitzen Klammern bezeichnet.
\index{g@$\langle g\rangle$}%
\end{definition}
\begin{beispiel}
  Um die Syntax zu verstehen, betrachten wir eine durch \(a\) erzeugte Gruppe \(G = \langle a \rangle\).
  Das bedeutet, dass \(G\) die Elemente \(a, aa, aaa, \ldots\) sowie \(a^{-1}, a^{-1}a^{-1}, \ldots\) und ein neutrales Element \(\mathds{1} = aa^{-1}\) enthält.
\end{beispiel}
\begin{beispiel}
  Als anschaulicheres Beispiel können wir eine zyklische Untergruppe des \(n\)-Gon formalisieren.
\index{n-Gon@$n$-Gon}%
  Wir bezeichnen mit \(r\) eine Drehung im Gegenuhrzeigersinn von \(360^\circ/n\) um einen Punkt.
  Diese Definition reicht aus, um die gesamte Symmetriegruppe
  \[
    C_n = \langle r \rangle
      = \{\mathds{1}, r, r^2, \ldots, r^{n-1}\}
  \]
\index{Cn@$C_n$}%
  der Drehungen eines \(n\)-Gons zu erzeugen.
  Das liegt daran, dass wir durch die mehrfache Verwendung von \(r\) jeden Winkel erzeugen k\"onnen, der die Rotationssymmetrie bewahrt.
  In ähnlicher Weise, aber weniger interessant, enthält die Reflexionssymmetriegruppe \(\langle\sigma\rangle\) nur \(\left\{\mathds{1}, \sigma\right\}\), weil \(\sigma^2 = \mathds{1}\).
\end{beispiel}

Wenn wir diese Idee nun erweitern, können wir mit einem Erzeugendensystem
komplexere Strukturen aufbauen.

%@Naoki Are you ok with my grammar fixes I'm not 101% shore how to use the word Erzeugendensystem? 
\begin{definition}[Erzeugendensystem]
  Jede diskrete Gruppe kann durch eines oder mehrere ihrer Elemente generiert werden.
  Wir lassen \(g_1, g_2, g_3, \ldots\) erzeugenden Elemente einer Symmetriegruppe sein.
  Da es mehrere Erzeuger gibt, müssen auch die sogenannten Definitionsgleichungen gegeben werden, die die Multiplikationstabelle vollständig definieren.
  Die Gleichungen sind ebenfalls in den Klammern angegeben.
  Die erzeugenden Elementen bauen zusammen mit den Definitionsgleichungen ein Erzeugendensystem.
\end{definition}
\begin{beispiel}
  Wir werden nun alle Symmetrien eines \(n\)-Gons beschreiben, was bedeutet, dass wir die Operationen \(r\) und \(\sigma\) kombinieren.
  Die Definitionsgleichungen sind \(r^n = \mathds{1}\), \(\sigma^2 = \mathds{1}\) und \((\sigma r)^2 = \mathds{1}\).
  Die ersten beiden sind ziemlich offensichtlich.
  Die letzte wird oft auch als Inversion bezeichnet, weil die Anwendung von \(\sigma r\) dasselbe ist wie das Ziehen einer Linie von einem Punkt, die durch den Ursprung geht, und das Verschieben des Punktes auf die andere Seite des Nullpunkts.
  Wenn man dies zweimal macht, geht man zurück zum Anfangspunkt.
  Daraus ergibt sich die so genannte Diedergruppe 
  \begin{align*}
    D_n &= \langle r, \sigma : r^n = \sigma^2 = (\sigma r)^2 = \mathds{1} \rangle \\
      &= \{
          \mathds{1}, r, \ldots, r^{n-1}, \sigma, \sigma r, \ldots, \sigma r^{n-1}
      \}. \qedhere
  \end{align*}
\end{beispiel}

Die Symmetrieoperationen, die wir bis jetzt besprochen haben, haben immer mindestens einen Punkt gehabt, der wieder auf sich selbst abgebildet wird.
Im Fall der Rotation war es der Drehpunkt, bei der Spiegelung die Punkte der Spiegelachse.
Dies ist jedoch keine Voraussetzung für eine Symmetrie, da es Symmetrien gibt, die jeden Punkt zu einem anderen Punkt verschieben können.
 Diesen Spezialfall, bei dem immer mindestens ein Punkt unverändert bleibt, nennt man Punktsymmetrie.
\begin{definition}[Punktgruppe]
  Wenn es einen Punkt gibt, der von jeder Gruppenoperation unverändert gelassen wird, ist die Symmetriegruppe eine Punktgruppe.
\end{definition}

\subsection{Algebraische Symmetrien}
Wir haben nun unseren Operationen Symbole gegeben, mit denen es tatsächlich möglich ist, Gleichungen zu schreiben.
Die anschliessende Frage ist dann, ob wir bereits mathematische Objekte haben, mit denen wir Gleichungen schreiben, die sich auf die gleiche Weise verhalten.
Die Antwort lautet natürlich ja.
Um es formaler zu beschreiben, werden wir einige Begriffe einführen.
\begin{definition}[Gruppenhomomorphismus]
\index{Gruppenhomomorphismus}%
  \(G\) und \(H\) seien  Gruppen mit unterschiedlichen Operationen \(\diamond\) bzw.
  \(\star\).
  Ein Homomorphismus\footnote{ Für eine ausführlichere Diskussion siehe \S\ref{buch:grundlagen:subsection:gruppen} im Buch.} ist eine Funktion \(f: G \to H\), so dass für jedes \(a, b \in G\) gilt \(f(a\diamond b) = f(a) \star f(b)\).
\index{Homomorphismus}%
  Man sagt, dass der Homomorphismus \(f\) \(G\) in \(H\) transformiert.
\end{definition}
\begin{beispiel}
  Die Rotationssymmetrie des Kreises \(C_\infty\), mit einem unendlichen Kontinuum von Werten \(\alpha \in \mathbb{R}\), entspricht genau dem komplexen Einheitskreis.
\index{Cunendlich@$C_{\infty}$}%
  Der Homomorphismus \(\varphi: C_\infty \to \mathbb{C}\) ist durch die Eulersche Formel \(\varphi(r) = e^{i\alpha}\) gegeben.
\end{beispiel}

\begin{definition}[Darstellung einer Gruppe]
  Die Darstellung einer Gruppe ist ein Homomorphismus
  \[
    \Phi: G \to \operatorname{GL}_n(\mathbb{R}),
  \]
  der eine Symmetriegruppe auf eine Menge von Matrizen abbildet.
  Äquivalent kann man sagen, dass ein Element aus der Symmetriegruppe auf einen Vektorraum \(V\) wirkt, indem man \(\Phi : G \times V \to V\) definiert.
\end{definition}
\begin{beispiel}
  Die Elemente \(r^k \in C_n\), wobei \(0 < k < n\), stellen abstrakt eine Drehung von \(2\pi k/n\) um den Ursprung dar.
  Die mit der Matrix 
  \[
    \Phi(r^k) = \begin{pmatrix}
      \cos(2\pi k/n) & -\sin(2\pi k/n) \\
      \sin(2\pi k/n) &  \cos(2\pi k/n)
    \end{pmatrix}
  \]
  definierte Funktion von \(C_n\) nach \(\operatorname{O}(2)\) ist eine Darstellung von \(C_n\).
  In diesem Fall ist die erste Gruppenoperation die Komposition und die zweite die Matrixmultiplikation.
  Man kann überprüfen, dass \(\Phi(r^2 \circ r) = \Phi(r^2)\Phi(r)\).
\end{beispiel}
