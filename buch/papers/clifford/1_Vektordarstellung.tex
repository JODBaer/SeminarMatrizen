\section{Vektoroperationen\label{clifford:section:Vektoroperationen}}
\rhead{Vektoroperationen}
Das grundsätzliche Ziel der geometrischen Algebra ist, die lineare Algebra zu einer Algebra mit Multiplikation zu erweitern und dieses Produkt dann geometrisch interpretieren, um  geometrische Probleme lösen zu können.
 \subsection{Vektordarstellung\label{clifford:section:Vektordarstellung}}
Vektoren können neben der üblichen Spaltendarstellung, auch als Linearkombination aus Basisvektoren
\begin{align*}
    \textbf{a} 
    &=
    \begin{pmatrix} 
    a_1 \\ a_2 \\ \vdots \\ a_n   
    \end{pmatrix} 
    =
    a_1 \begin{pmatrix}
    1 \\ 0 \\ \vdots \\ 0  
    \end{pmatrix} 
    + 
    a_2\begin{pmatrix} 
    0 \\ 1 \\ \vdots \\ 0  
    \end{pmatrix} + \dots 
    + 
    a_n\begin{pmatrix}
    0 \\ 0 \\ \vdots \\ 1  
    \end{pmatrix},\\
\intertext{oder auch als}
    &= 
    a_1\textbf{e}_1 
    +
    a_2\textbf{e}_2
    + 
    \dots + a_n\textbf{e}_n
    = 
    \sum_{i=1}^{n} a_i \textbf{e}_i
    \quad
    a_i \in \mathbb{R}
    , \textbf{e}_i \in \mathbb{R}^n
\end{align*}
dargestellt werden.
Diese Basisvektoren werden so gewählt, dass sie orthonormiert sind. 
\begin{beispiel}
Eine Linearkombination von Basisvektoren in $\mathbb{R}^4$ könnte wie folgt aussehen
    \begin{equation*}
        \begin{pmatrix} 
        42 \\ 2 \\ 1291 \\ 4    
        \end{pmatrix} 
        = 
        42 \begin{pmatrix}
        1 \\ 0 \\ 0 \\ 0 
        \end{pmatrix} 
        +
        2 \begin{pmatrix} 
        0 \\ 1 \\ 0 \\ 0 
        \end{pmatrix}
        +
        1291 
        \begin{pmatrix} 
        0 \\ 0 \\ 1 \\ 0 
        \end{pmatrix} 
        +
        4 \begin{pmatrix} 
        0 \\ 0 \\ 0 \\ 1 
        \end{pmatrix} 
        = 
        42\textbf{e}_1
        + 
        2\textbf{e}_2
        + 
        1291\textbf{e}_3
        + 
        4\textbf{e}_4.
    \end{equation*}
Dieses Beispiel ist für einen vierdimensionalen Vektor, dies kann selbstverständlich für beliebig viele Dimensionen nach demselben Schema erweitert werden.
\end{beispiel}
