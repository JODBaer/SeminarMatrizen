\section{Skalare, Vektoren, Matrizen und Tensoren\label{spannung:section:Skalare,_Vektoren,_Matrizen_und_Tensoren}}
\rhead{Skalare, Vektoren, Matrizen und Tensoren}
Der Begriff Tensor kann als Überbegriff der mathematischen Objekte Skalar, Vektor und Matrix betrachtet werden.
\index{Tensor}%
Allerdings sind noch höhere Stufen dieser Objekte beinhaltet.
Skalare, Vektoren oder Matrizen sind daher auch Tensoren.
Ein Skalar ist ein Tensor 0. Stufe.
\index{Stufe}%
Mit einem Vektor können mehrere Skalare auf einmal beschrieben werden.
Ein Vektor hat daher die Stufe 1 und ist höherstufiger als ein Skalar.
Mit einer Matrix können wiederum mehrere Vektoren auf einmal beschrieben werden.
Eine Matrix hat daher die Stufe 2 und ist noch höherstufiger als ein Vektor.
Versteht man diese Stufen, so versteht man den Sinn des Begriffs Tensor.

Jede Stufe von Tensoren verlangt andere Rechenregeln.
So zeigt sich auch der Nachteil von Tensoren mit Stufen höher als 2.
Man ist also bestrebt höherstufige Tensoren mit Skalaren, Vektoren oder Matrizen zu beschreiben.

In den 40er Jahren des 19.~Jahrhunderts wurde der Begriff Tensor von William Rowan Hamilton in die Mathematik eingeführt.
\index{Hamilton, William Rowan}%
James Clerk Maxwell hat bereits mit Tensoren operiert, ohne den Begriff Tensor gekannt zu haben.
\index{Maxwell, James Clerk}%
Erst Woldemar Voigt hat den Begriff in die moderne Bedeutung von Skalar, Matrix und Vektor verallgemeinert.
\index{Voigt, Woldemar}
Er hat in der Elastizitätstheorie als erstes Tensoren eingesetzt und beschrieben.
\index{Elastizitätstheorie}%
Auch Albert Einstein hat solche Tensoren eingesetzt,
\index{Einstein, Albert}%
um in der Relativitätstheorie die Änderung der vierdimensionalen Raumzeit beschreiben zu können
\index{Relativitätstheorie}%
\index{Raumzeit}%
\cite{spannung:Tensor}.

