\section{Der Spannungszustand\label{spannung:section:Der Spannungsustand}}
\rhead{Der Spannungszustand}
Ein Spannungszustand ist durch alle Spannungen, welche in einem beliebigen Punkt im Körper wirken, definiert (siehe Abbildung~\ref{fig:Bild2}).
Änderungen der äusseren Kräfte verändern die inneren Spannungszustände im Material.
Um alle Spannungen eines Punktes darstellen zu können,
stellt man sich modellhaft ein infinitesimales Bodenelement in Form eines Würfels vor.
Man spricht auch von einem Elementarwürfel.

\begin{figure}
	\centering
	\includegraphics[width=0.4\linewidth,keepaspectratio]{papers/spannung/Grafiken/Bild2.png}
	\caption{Infinitesimales Bodenelement mit den neun Spannungen}
	\label{fig:Bild2}
\end{figure}

Es werden jeweils drei Seiten dieses Würfels betrachtet, wobei die drei gegenüberliegenden Seiten im Betrag die selben Spannungen aufweisen,
sodass der Elementarwürfel im Gleichgewicht ist.
Wäre dieses Gleichgewicht nicht vorhanden, käme es zu Verschiebungen und Drehungen.
Das infinitesimale Bodenteilchen hat die Koordinatenachsen $1$, $2$, $3$.
\index{Normalspannung}%
\index{Schubspannung}%
Veränderungen der Normalspannungen können durch Schubspannungen kompensiert werden und umgekehrt.
So sind insgesamt neun verschiedene Spannungen möglich, konkret sind dies drei Normal- und sechs Schubspannungen.
Normalspannungen wirken normal (mit rechtem Winkel) zur angreifenden Fläche und Schubspannungen parallel zur angreifenden Fläche.
Alle Beträge dieser neun Spannungen am Elementarwürfel bilden den Spannungszustand.
Daraus können die äquivalenten Dehnungen $\varepsilon$ mit Hilfe des Hook'schen Gesetzes berechnet werden.
Daher gibt es auch den entsprechenden Dehnungszustand.


\section{Einachsiger Spannungszustand\label{spannung:section:Spannungsustand}}
\rhead{Einachsiger Spannungszustand}

Im einachsigen Spannungszustand herrscht nur die Normalspannung $\sigma_{11}$ (siehe Abbildung~\ref{fig:Bild1}).
Das Hook'sche Gesetz beschreibt genau diesen 1D Spannungszustand.
Nach Hooke gilt:
\[
F
\sim
\Delta l
.
\]
Teilt man beide Seiten durch die Konstanten $A$ und $l_0$, erhält man
\[
\frac{F}{A}
=
\sigma
\sim
\varepsilon
=
\frac{\Delta l}{l_0}
\]
und somit
\[
\sigma
\sim
\varepsilon
,
\]
mit
\begin{align*}
	l_0 &= \text{Länge zu Beginn [\si{\meter}]} \\
	  A &= \text{Fläche [\si{\meter\squared}].}
\end{align*}
Diese Beziehung gilt bei linear-elastischen Materialien, welche reversible Verformungen zulassen.
Es ist praktisch, die relative Dehnung $\varepsilon$ anzugeben und nicht eine absolute Längenänderung $\Delta l$.
\index{Dehnung, relativ}%
\index{Längenänderung}%
\index{Elastizitätsmodul}%
\begin{figure}
	\centering
	\includegraphics[width=0.35\linewidth,keepaspectratio]{papers/spannung/Grafiken/Bild1.png}
	\caption{1D Spannungszustand aus einer quaderförmigen Bodenprobe}
	\label{fig:Bild1}
\end{figure}
Mithilfe vom Elastizitätsmodul $E$ (auch Youngscher Modul) als Proportionalitätskonstante lässt sich der eindimensionale Fall mit
\index{Youngscher Modul}
\[
\sigma
=
E\cdot\varepsilon
\]
beschreiben.
Im Falle, dass $E$ nicht konstant ist, wird dieser durch
\[
E
=
\frac{d\sigma}{d\varepsilon}
\]
ausgedrückt.
